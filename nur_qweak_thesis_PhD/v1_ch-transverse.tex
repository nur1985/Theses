%\chapter{Beam Normal Single Spin Asymmetry in Inelastic Scattering}
\chapter{BEAM NORMAL SINGLE SPIN ASYMMETRY}
\label{BEAM NORMAL SINGLE SPIN ASYMMETRY}

%%%%%%%%%%%%%%%%%%%%%%%%%%%%%%%%%%%%%%%%%%%%%%%%%%%%%%%%%%%%%
\section{Introduction}
\label{Introduction}
%The objective of the Q-weak experiment is to measure the parity violating asymmetry ($\sim$250~ppb) in elastic electron-proton (e+p) scattering to determine the proton's weak charge with an uncertainty of $\sim$ 4\% combined statistical and systematic errors~\cite{qweak_proposal_2007}. There is a parity conserving Beam Normal Single Spin Asymmetry (BNSSA) or transverse asymmetry ($A_{N}$) on H$_{2}$ with a $\sin(\phi)$-like dependence due to 2-$\gamma$ exchange. The size of $A_{N}$ is few ppm, so a few percent residual transverse polarization in the beam, in addition to potentially small broken azimuthal asymmetries in the main detector, might lead to few ppb corrections to the Q-weak data. As part of a program of $A_{N}$ background studies, the first measurement of $A_{N}$ in the N-to-$\Delta$ transition was performed using the Q-weak apparatus. $A_{N}$ from electron-nucleon scattering is also a unique tool to study the $\gamma^{*}\Delta\Delta$ form factors~\cite{Alexandrou:2009hs}. This chapter contains the details of the analysis on transverse asymmetry in  N-to-$\Delta$ region for H$_{2}$ target.
Dedicated measurements of the beam normal single spin asymmetry in inelastic e+p, and e+N scattering with $\Delta$(1232) in the final state were performed during 18th - 20th February 2012 at Hall-C of Jefferson Lab using Q-weak apparatus.


%%%%%%%%%%%%%%%%%%%%%%%%%%%%%%%%%%%%%%%%%%%%%%%%%%%%%%%%%%%%%
\section{Experimental Method}
\label{Experimental Method}

The Q-weak longitudinal measurement setup~\cite{qweak_proposal_2007} was used for inelastic transverse measurement. The electron beam polarization was changed from nominal longitudinal setup to produce fully horizontal/ vertical polarization using the double Wien filter at the injector (Section~\ref{Polarized Source and Helicity Reversal}). Torodial magnet setting was changed to 6700 A (nominal magnet current for elastic running was 8901 A) to focus inelastically scattered electron into the main \v{C}erenkov detector. 
%The condition of the experimental setup is described in the next section. 


\subsection{Available Data Set and Condition of Experimental Data Taking}
\label{Available Data Set and Condition of Experimental Data Taking}

%A dedicated measurement of the beam normal single spin asymmetry in inelastic e+p, and e+N scattering with $\Delta$(1232) in the final state were performed during 18th - 20th February 2012 at Hall-C of Jefferson Lab using Q-weak apparatus. 
Total collected data after hardware and software quality checks is shown in Table~\ref{tab:transverse_inelastic_data_set}. The QTor current of 6700~A selects the inelastic events in the e+P and e+N scattering. Data on both sides of the inelastic peak (6000~A and 7300~A) were taken to improve simulation of elastic dilution. Two transverse spin orientations, horizontal and vertical, were used to study the asymmetry cancellation between the opposite octants (octants 3 \& 7 and 1 \& 5 for horizontal and vertical respectively) of the \v{C}erenkov detectors. Data were collected on liquid hydrogen (LH$_{2}$) cell, 4\% thick downstream aluminum alloy (Al), and a 1.6\% thick downstream carbon foil ($^{12}$C) with 1.155~GeV beam for both spin orientations (as shown in Table~\ref{tab:transverse_inelastic_data_set}). 
%Vertical transverse are shown in parenthesis, rest are for horizontal transverse.
Different beam currents ($I$) were used on different targets, as shown in Table~\ref{tab:transverse_inelastic_data_set}.
The beam was rastered on the target over an area of 4~mm$\times$4~mm by the fast raster system to minimize the target boiling.
The Insertable Half Wave Plate (IHWP) was used to remove further helicity correlated beam asymmetries and was reversed at intervals of about 2~hours. 
More information about the condition of data taking is given in APPENDIX-\ref{Beam Normal Single Spin Asymmetry in Inelastic e-p Scattering} section~\ref{Condition of Experimental Data Taking}.
%Q-weak also performed the first known measurements of the inelastic beam normal single spin asymmetries with a (1232) in the final state from all of these targets. 


%\renewcommand\arraystretch{2.4} 
%\setlength\minrowclearance{2.4pt}
\renewcommand{\arraystretch}{1.0} % make cell wider
\begin{table}[!h]
 \begin{center}
   \caption
	[Transverse N-to-$\Delta$ data set.]   
   {Transverse N-to-$\Delta$ data set. The data set for vertical transverse polarization are in parentheses, rest are from horizontal transverse polarization. The beam current for different targets are shown in second last row. Amount of transverse data collected in terms of the total charge in Coulombs are shown in bottom row.}
  \begin{tabular}{ c | c | c  c  c | c  c }
%    \hline
    \noalign{\hrule height 1pt}
%    \multirow{2}{*}{HWP} & QTor current  & & QTor current & &  \multicolumn{2}{c}{QTor current} \\
    \multirow{3}{*}{IHWP} & \multicolumn{6}{c}{QTor current} \\ \cline{2-7}
%	\cline{2-7}%\hline
		 &  6000 A & & 6700 A & &  \multicolumn{2}{c}{7300 A}\\
	\cline{2-7}%\hline
	     & LH$_{2}$ & LH$^{\dagger}_{2}$ & Al$^{\dagger\dagger}$ & $^{12}$C &  LH$_{2}$ & Al \\
%	\cline{2-7}%\hline
    \noalign{\hrule height 1pt}
%    \hline
	IN  & \pbox{3cm}{16152\\ 16153} & \pbox{3cm}{(16066)\\ 16131\\ 16132} & \pbox{3cm}{(16067)\\ 16115\\ 16116} & \pbox{3cm}{16150\\ 16151} & \pbox{3cm}{16133\\ 16134\\ 16135} & \pbox[c][2cm][c]{3cm}{ 16122\\ 16123\\ 16124\\ 16160 } \\ 
    \noalign{\hrule height 1pt}
%	\hline
	OUT & \pbox{5cm}{16154\\ 16156\\ 16157\\ 16158} & \pbox{5cm}{(16065)\\ 16129\\ 16130} & \pbox{5cm}{\vspace{5pt}(16068)\\ (16069)\\ 16117\\ 16118\\ 16119\vspace{5pt}} & \pbox{5cm}{16148\\ 16149} & \pbox{5cm}{16136\\ 16137} & \pbox{5cm}{16120\\ 16121\\ 16161} \\ 
%    \hline
    \noalign{\hrule height 1pt}
	   Beam current $I$ [$\mu$A] & 180 & 180 & 60 & 75 & 180 & 60 \\
    \noalign{\hrule height 1pt}
%	\hline
	   Collected Data [C] & 1.5 & 1.8 (1.9) & 0.8(0.4) & 0.6 & 2.0 & 0.9 \\
%	\hline
    \noalign{\hrule height 1pt}
   \end{tabular}
 \label{tab:transverse_inelastic_data_set}
 \end{center}
\end{table}
\renewcommand{\arraystretch}{1.0} % make cell wider

%\rule{1pt}{4cm}

In this dissertation, full analysis of the beam normal single spin asymmetry from inelastic electron-proton scattering on LH$_{2}$ target, indicated by $\dagger$ in Table~\ref{tab:transverse_inelastic_data_set}, will be discussed. The transverse asymmetry on Al target, indicated by $\dagger\dagger$ in the table, was also analyzed as a background correction for the LH$_{2}$ target. The analysis of the remaining data are ongoing and will not be covered in this dissertation.

%%%%%%%%%%%%%%%%%%%%%%%%%%%%%%%%%%%%%%%%%%%%%%%%%%%%%%%%%%%%%
\section{Extraction of Raw Asymmetries}
\label{Extraction of Raw Asymmetries}

Single detector asymmetry was obtained by averaging two PMT asymmetries of the \v{C}erenkov detector. The error weighted average of the asymmetries from runlets, $\sim$5 minutes long data samples, was extracted as the average asymmetry for a given data set. To extract the uncorrected raw asymmetry $A_{raw}$ from the detectors, the average asymmetry for the two different Insertable Half Wave Plate (IHWP) settings, IN and OUT, were determined separately for each main detector bar. The asymmetries measured in the IHWP configurations were sign corrected for the extra spin flip and averaged together after checking for the IHWP cancellation of the false asymmetries. The error weighted value of IN-OUT yields the measured raw asymmetry for each bar. These raw asymmetries were then plotted against the detector octant number, which represents the location of the detector in the azimuthal plane ($\phi$ = (octant - 1)$\times$45$^{\circ}$), and they were fitted using a function of the form in Equation~\ref{equ:eqFitTransverse}.
The transverse asymmetries show azimuthal modulation behavior, hence this analysis will focus on the azimuthal dependence of the detector asymmetries.

\begin{equation} \label{equ:eqFitTransverse}
f(\phi) = \left\lbrace 
\begin{aligned}
\text{Horizontal transverse:}~A_{M}^{H} \sin(\phi + \phi_{0}^{H}) + C^{H}\\
\text{Vertical transverse:}~A_{M}^{V} \cos(\phi + \phi_{0}^{V}) + C^{V}
\end{aligned}
\right.
\end{equation}

Here, $\phi$ is the azimuthal angle in the transverse plane to the beam direction. $\phi$ = 0 indicates beam left, $\phi_{0}$ is the constant phase offset in $\phi$, $A_{M}$ is the measured asymmetry (amplitude) of the azimuthal modulation generated by BNSSA, and $C$ is a constant appearing for monopole asymmetries such as the parity violating asymmetry generated by residual longitudinal polarization in the beam. The measured un-regressed raw asymmetries for the horizontal and vertical transverse polarization on LH$_{2}$ target are $A_{raw}^{H}$ = 5.34~$\pm$~0.53~ppm and $A_{raw}^{V}$ = 4.60~$\pm$~0.81~ppm respectively. 

%\begin{center}
%\framebox[\frameboxsize][c]{Combined (error weighted) asymmetry $A_{M}^{in}$ is 5.095$\pm$0.444~ppm.}
%\end{center}


\begin{figure}[!h]
	\begin{center}
	\includegraphics[width=15.0cm]{figures/MD_v_transverse_5+1_Sensitivities}
	\end{center}
	\caption
	[Azimuthal dependence of the main detector sensitivities to HCBA in the vertical LH$_{2}$ transverse data set.]	
	{Azimuthal dependence of the main detector sensitivities to HCBA with respect to 5+1 regression scheme in the vertical LH$_{2}$ transverse data set are shown here. Beam positions and angles have sinusoidal dependence with octant. No such dependence is seen for energy and charge. Two IHWP states are shown separately for each beam parameter. Fit functions used to fit the parameters are shown on the plot. The constant in the fit gives the error weighted average of the sensitivities. See APPENDIX-\ref{Beam Normal Single Spin Asymmetry in Inelastic e-p Scattering}, section~\ref{Corrections}  for the sensitivities and corrections from full data sets.}
	\label{fig:MD_v_transverse_5+1_Sensitivities}
\end{figure}


%%%%%%%%%%%%%%%%%%%%%%%%%%%%%%%%%%%%%%%%%%%%%%%%%%%%%%%%%%%%%
\section{Asymmetry Correction using Linear Regression}
\label{Asymmetry Correction using Linear Regression}

The helicity correlated changes in the electron beam position, angle, and energy change the effective scattered angle and energy of the electrons in the detector acceptance.
Changes in these beam parameters can create false asymmetry in the detector and need to be corrected before the extraction of the physics asymmetry. A multi variable linear regression~\cite{linRegTechNote} is used to remove the beam asymmetries from the raw \v{C}erenkov detector asymmetries as shown in Equation~\ref{equ:regression}.

\begin{equation} \label{equ:regression}
A_{M} = A_{raw} - \sum^{6}_{i=1} \left(\frac{\partial A_{raw} }{ \partial T_{i} }\right) \Delta T_{i}
\end{equation}

Here $A_{M}$ is the measured asymmetry after regression, and ($\partial A_{raw}/\partial T_{i}$) is the detector sensitivity to a helicity-correlated beam parameter $T_{i}$ with differences $\Delta T_{i}$. During this measurement period, the helicity-correlated differences were stable (shown in Figure~\ref{fig:transverse_LH2_h_diff}, and~\ref{fig:transverse_LH2_v_diff}) and are summarized in Table~\ref{tab:differences}. The detector sensitivity slopes are calculated with linear regression, which uses natural beam motion during a runlet and considers correlations between different beam parameters. The asymmetries presented in this dissertation are regressed against six (5+1) beam parameters ($T_{i}$): horizontal position ($X$), horizontal angle ($X^{\prime}$), vertical position ($Y$), vertical angle ($Y^{\prime}$), the energy asymmetry ($A_{E}$), and the charge asymmetry ($A_{Q}$). Regression assumes a linear correlation between each variable. The sensitivities of the \v{C}erenkov detectors to different helicity correlated beam parameters have azimuthal dependence, as shown in Figure~\ref{fig:MD_v_transverse_5+1_Sensitivities} (shown for vertical transverse data only, horizontal transverse can be found in Figure~\ref{fig:MD_h_transverse_5+1_Sensitivities}). This azimuthal dependence of the position and angle sensitivities are a result of the movement of the scattered electron profile across the octants which changes the effective scattering angle of the detected electrons not specific to the transverse asymmetry measurement. The position and angle sensitivities are anti-correlated. The energy and charge sensitivities are not expected to have any azimuthal dependence since they do not change the acceptance.
The size of the applied correction to the raw asymmetries depends on the size of the helicity-correlated beam parameter differences $\Delta T_{i}$ and the sensitivities ($\partial A_{raw}/\partial T_{i}$). The size of the corrections were $\sim$2-3 order of magnitude smaller compared to the size of the measured asymmetry and are shown in Figure~\ref{fig:MD_v_transverse_5+1_corrections} (shown for vertical transverse data only, horizontal transverse can be found in Figure~\ref{fig:MD_h_transverse_5+1_corrections}). The total applied regression correction (Figure~\ref{fig:MD_v_transverse_5+1_TotalCorrections}) is dominated by the $X$ correction (Figure~\ref{fig:MD_v_transverse_5+1_corrections} top left).

\begin{table}[!h]
\begin{center}
  	\caption
	[Beam parameter differences during for the horizontal and vertical transverse data set.]  	
  	{Beam parameter differences during for the horizontal and vertical transverse data set. The X differences are higher compared to Y differences.}
  \begin{tabular}{ c | c  c | c  c }
%	\hline
    \noalign{\hrule height 1pt}
    Beam parameter & \multicolumn{2}{c|}{Horizontal} & \multicolumn{2}{c}{Vertical} \\ 
    \cline{2-5}
%	\hline
    	differences &	IHWP IN	&	IHWP OUT &	 IHWP IN	&	IHWP OUT  \\
%	\hline
    \noalign{\hrule height 1pt}
	$\Delta$X~[nm] & 23.8~$\pm$~2.1 & 20.6~$\pm$~2.3 & 15.4~$\pm$~3.1	& 58.0~$\pm$~3.6\\
	$\Delta$Y~[nm]	& 6.9~$\pm$~2.1 & 5.6~$\pm$~2.3 & 20.2~$\pm$~3.1 & 15.4~$\pm$~3.6 \\
	$\Delta$X$^{\prime}$~[nrad] & 0.7~$\pm$~0.1 & 0.7~$\pm$~0.1 & 0.6~$\pm$~0.2	& 1.3~$\pm$~0.2\\
	$\Delta$Y$^{\prime}$~[nrad] & 0.2~$\pm$~0.1 & -0.3~$\pm$~0.1 & 0.6~$\pm$~0.2	& 0.9~$\pm$~0.2\\
	$\Delta$E~[ppb]	& -2.3~$\pm$~2.1 & -1.5~$\pm$~2.3 & 0.5~$\pm$~3.1	& -5.4~$\pm$~3.6\\
	$\Delta$A$_{Q}$~[ppb]	& 8.2~$\pm$~0.5 & -237.3~$\pm$~55.6 & 60.1~$\pm$~0.7	& 158.1~$\pm$~88.1\\
%	\hline
    \noalign{\hrule height 1pt}
  	\end{tabular}
  \label{tab:differences}
\end{center}
\end{table}

\begin{figure}[!h]
	\begin{center}
	\includegraphics[width=15.0cm]{figures/MD_v_transverse_5+1_corrections}
	\end{center}
	\caption
	[Main detector corrections vs octant for vertical transverse data set.]	
	{Main detector corrections (using sensitivities and differences from 5+1 regression scheme) vs octant for vertical LH$_{2}$ transverse data set are shown here. Beam positions and angles have sinusoidal dependence with octant inherited from the sensitivities. No such dependence is seen for energy and charge. Two IHWP states are shown separately for each beam parameter.}
	\label{fig:MD_v_transverse_5+1_corrections}
\end{figure}

\begin{figure}[!h]
	\begin{center}
	\includegraphics[width=15.0cm]{figures/MD_v_transverse_5+1_TotalCorrections}
	\end{center}
	\caption
	[Total corrections vs octant for vertical transverse data set.]	
	{Total corrections (using sensitivities and differences from 5+1 regression scheme) vs octant for vertical transverse data set are shown here. Total correction is the sum of all the corrections (with sign) shown in Figure~\ref{fig:MD_v_transverse_5+1_corrections}.}
	\label{fig:MD_v_transverse_5+1_TotalCorrections}
\end{figure}



%\begin{table}[!h]
%\begin{center}
%  	\caption
%	[Beam parameter differences during for the horizontal and vertical transverse data set.]  	
%  	{Beam parameter differences during for the horizontal and vertical transverse data set.}
%  \begin{tabular}{ c | c  c | c  c }
%%	\hline
%    \noalign{\hrule height 1pt}
%         \multirow{2}{*}{Beam parameter} & \multicolumn{2}{c|}{Horizontal} & \multicolumn{2}{c}{Vertical} \\ 
%     \cline{2-5}
%%	\hline
%    	&	IHWP IN	&	IHWP OUT &	 IHWP IN	&	IHWP OUT  \\
%%	\hline
%    \noalign{\hrule height 1pt}
%	Target X position differences $\Delta$X~[nm] & 23.8~$\pm$~2.1 & 20.6~$\pm$~2.3 & 15.4~$\pm$~3.1	& 58.0~$\pm$~3.6\\
%	Target Y position differences $\Delta$Y~[nm]	& 6.9~$\pm$~2.1 & 5.6~$\pm$~2.3 & 20.2~$\pm$~3.1 & 15.4~$\pm$~3.6 \\
%	Target X angle differences $\Delta$X$^{\prime}$~[nrad] & 0.7~$\pm$~0.1 & 0.7~$\pm$~0.1 & 0.6~$\pm$~0.2	& 1.3~$\pm$~0.2\\
%	Target Y angle differences $\Delta$Y$^{\prime}$~[nrad] & 0.2~$\pm$~0.1 & -0.3~$\pm$~0.1 & 0.6~$\pm$~0.2	& 0.9~$\pm$~0.2\\
%	Energy differences $\Delta$E~[ppb]	& -2.3~$\pm$~2.1 & -1.5~$\pm$~2.3 & 0.5~$\pm$~3.1	& -5.4~$\pm$~3.6\\
%	Charge asymmetry $\Delta$A$_{Q}$~[ppb]	& 8.2~$\pm$~0.5 & -237.3~$\pm$~55.6 & 60.1~$\pm$~0.7	& 158.1~$\pm$~88.1\\
%%	\hline
%    \noalign{\hrule height 1pt}
%  	\end{tabular}
%  \label{tab:differences}
%\end{center}
%\end{table}

\begin{figure}[!h]
	\begin{center}
	\includegraphics[width=15.0cm]{figures/asymmetry_In_Out_H2}
	\end{center}
	\caption
	[Main detector asymmetry for horizontal, vertical transverse data set.]
	{Main detector asymmetry for horizontal (top), vertical (bottom) transverse data set. For comparison, asymmetries for IN and OUT data are also shown separately. The regressed asymmetries change sign with the insertion of the IHWP with comparable amplitudes due to spin dependence. The (IN+OUT)/2 asymmetries of the eight \v{C}erenkov detectors, given by $C^\textrm{(In+Out)/2}$ is compatible with zero except in the vertical data set. The extraction of BNSSA depends on the amplitudes in the fits and by comparison of IN and OUT, not the constant term.}
	\label{fig:asymmetry_In_Out_H2}
\end{figure}



The regressed (5+1) asymmetries measured using horizontal and vertical transverse polarization beam on LH$_{2}$ target are shown in Figure~\ref{fig:asymmetry_In_Out_H2}. The azimuthal modulating asymmetry flips sign with the insertion of the IHWP as expected. The vertical transverse asymmetries may show sign of phase shift between IHWP IN and IHWP OUT settings, but may be explained due to statistical fluctuation. Transverse polarization angle was $\sim$3-4\degrees{} off from ideal settings during the measurement, which can not be confirmed with the statistics in hand. The (IN+OUT)/2 given by the $C^{(IN+OUT)/2}$ are compatible with zero  within the measurement uncertainties. This null asymmetry indicates the azimuthal modulating component in both IHWP IN and OUT are analogous, and the non-polarization dependent false beam asymmetries were successfully removed by the regression.

The error weighted value of IN-OUT yields the measured regressed asymmetry for each bar. As expected from the azimuthal dependence of the BNSSA, there is a 90\degrees{} phase offset between the two modulations, as shown in Figure~\ref{fig:asymmetry_H2}. The measured regressed asymmetries using horizontal and vertical transverse polarization are extracted as $A_{M}^{H}$ = 5.34~$\pm$~0.53~ppm and  $A_{M}^{V}$ = 4.53~$\pm$~0.81~ppm respectively. The combined (error weighted average) regressed asymmetry from horizontal and vertical transverse polarization is given by

\begin{equation} \label{equ:asymmetryMeasured}
A_{M} = 5.095~\pm~0.444~\text{ppm (stat)}.
\end{equation}

This measurement provides a $\sim$9\% statistical measurement of the BNSSA in inelastic e+p scattering (not corrected for backgrounds, polarization or other experimental related systematic uncertainties). Regression has minimal effect on the extracted measured asymmetries as the corrections are too small compared to the size of the asymmetries.

\begin{figure}[!h]
	\begin{center}
	\includegraphics[width=15.0cm]{figures/asymmetry_H2}
	\end{center}
	\caption
	[Regressed main detector asymmetry for horizontal, vertical transverse data set on LH$_{2}$.]	
	{Regressed main detector asymmetry for horizontal, vertical transverse polarization are shown with red circle and blue square respectively. Data points for horizontal transverse are $\sim$4~hour long measurement, whereas vertical transverse data points are $\sim$2~hour long. The fit functions used are $A_{M}^{H} \sin(\phi + \phi_{0}^{H}) + C^{H}$ for horizontal transverse and $A_{M}^{V} \cos(\phi + \phi_{0}^{V}) + C^{V}$ for vertical transverse respectively. Asymmetries in each case shows $\sim$90\degrees{} phase offset, as expected between horizontal and vertical configurations.}
	\label{fig:asymmetry_H2}
\end{figure}

\subsection{Azimuthal Acceptance Correction}
\label{Azimuthal Acceptance Correction}

The acceptance of a single Q-weak \v{C}erenkov detector is only 49\% of an octant (section~\ref{Q-weak Kinematics}). So the reported asymmetry from a detector is an average over 22\degrees{} azimuthal angle ($\phi$). Each detector bar measures an average asymmetry over a range of $\phi$ selected by the collimators about its nominal location (details in~\cite{elog:birchall_analysis373}). 
%The method in~\cite{elog:birchall_analysis373} was used to take in to account the fact that each bar measures an average asymmetry over a range of $\phi$ selected by the collimator's about its nominal location. 
The effect of averaging cosines for a variable of the form $y(\phi) = A \cos(\phi + \delta)$ over the azimuthal angle yields 

\begin{equation} \label{equ:eqDetectorNonlinearity}
AVG[y(\phi)] = \frac{ A\int_{\phi_{0}-\Delta\phi}^{\phi_{0}+\Delta\phi} \cos(\phi+\delta)\,\mathrm{d}\phi }{(\phi_{0}+\Delta\phi)-(\phi_{0}-\Delta\phi)} = A \cos (\phi_{0} + \delta) \times \frac{\sin\Delta\phi}{\Delta\phi}, 
\end{equation}

where $\phi_{0}$ is the nominal azimuthal location of the detector with $\Delta\phi$ coverage. Similarly, for sines, $AVG[y(\phi)] = A \sin (\phi_{0} + \delta) \times \frac{\sin\Delta\phi}{\Delta\phi}$. So the measured asymmetry from each detector needs to be scaled by a factor of $\frac{\sin\Delta\phi}{\Delta\phi}$ to correct for the acceptance. 
A sinusoidal function was used to extract the transverse asymmetry due to azimuthal dependence of the measured asymmetry. So it was important to correct for the detector azimuthal acceptance. Assuming the collimator removes 49\% of the octant acceptance (i.e 49\% of 45\degrees{}), $\Delta\phi$ = 11.025\degrees{}
yields the scale factor to be $\frac{\sin\Delta\phi}{\Delta\phi}$ = 0.9938.
% implies the amplitude from the transverse fits to the octant asymmetries should be scaled by a factor of $\frac{\sin\Delta\phi}{\Delta\phi}$ = 0.9938. 
The detector acceptance corrected measured asymmetry can be extracted as

\begin{equation} \label{equ:asymmetryDetAcptCorrected}
A_{M}^{in} = 5.127~\text{ppm}.
\end{equation}

A conservative 50\% uncertainty was used for $\Delta\phi$, which yields a systematic uncertainty of 0.004 in the correction.

%%%%%%%%%%%%%%%%%%%%%%%%%%%%%%%%%%%%%%%%%%%%%%%%%%%%%%%%%%%%%
\section{Systematic Uncertainties}
\label{Systematic Uncertainties}
The dominant uncertainty in the measured asymmetry for this measurement is statistical (9\%). A preliminary treatment of the systematic uncertainty performed on the data set is presented in this section.

\subsection{Regression Scheme Dependence}
\label{Regression Scheme Dependence}

%As discussed in previous section, the regression has minimal impact on the measured asymmetry. 
%Several regression schemes were used in analyzing these data.
The 5+1 linear regression scheme considered for this analysis is one of the many different schemes available and was worth investigating the regression corrections from several choices for the regression basis. A list of all the independent variables for different regression sets are shown in APPENDIX-\ref{REGRESSION SCHEME}. Ideally, the regression results from all the schemes should agree if all equipment is functioning properly and the regression is being done properly. The differences in the regressed asymmetries can arise from differences in the noise, resolution, and non-linear response of the monitors. To compare for the systematic studies, a common set of event cuts~\cite{rakitha_qweak} are applied to all regression schemes to match the quartets used by each scheme.

\begin{table}[!h]
 \begin{center}
	\caption
	[Asymmetries from different regression scheme for horizontal and vertical transverse data set.]
	{Asymmetries from different regression scheme along with un-regressed asymmetry are shown for horizontal and vertical transverse data set from Run 2 Pass 5 database. Correction (difference between regressed and un-regressed asymmetry) on measured main detector transverse asymmetry due to regression schemes are small compared to the amplitude of the measured asymmetry. Set 5 and 6 were not available due to failure of BPM 9b during Run 2. Set 9 was ignored for this analysis as it contains the upstream luminosity monitor, which is not an independent variable (mainly used for diagnostic purpose), as one of the regression variable (more details about regression variables are in APPENDIX-\ref{REGRESSION SCHEME}).}
  \begin{tabular}{ c | c  c | c  c}
%    \hline
    \noalign{\hrule height 1pt}
     \multirow{3}{*}{Regression scheme} & \multicolumn{2}{c|}{Horizontal} & \multicolumn{2}{c}{Vertical} \\ 
     \cline{2-5}
     & Asymmetry & Correction & Asymmetry & Correction\\
	& [ppm]  & [ppm] & [ppm]  & [ppm] \\
%	\hline
    \noalign{\hrule height 1pt}
	UnReg	&	5.339	&	0.000	& 4.602	&	0.000	\\
	std		&	5.343	&	0.004	& 4.524	&	-0.078	\\
	5+1		&	5.343	&	0.004	& 4.525	&	-0.077	\\
	set3		&	5.343	&	0.004	& 4.525	&	-0.077	\\
	set4		&	5.343	&	0.004	& 4.527	&	-0.076	\\
	set7		&	5.347	&	0.007	& 4.529	&	-0.073	\\
	set8		&	5.346	&	0.007	& 4.531	&	-0.072	\\
\st{set9}	&	\st{5.343}	&	\st{0.003}	& \st{4.534}	&	\st{-0.069}	\\
	set10	&	5.343	&	0.003	& 4.526	&	-0.077	\\
	set11	&	5.343	&	0.004	& 4.524	&	-0.078	\\
%	\hline
    \noalign{\hrule height 1pt}
	Max - Min	&	set8 - set10 & 0.004	& set8 - set11	&	0.006	\\		
%	\hline
    \noalign{\hrule height 1pt}
   \end{tabular}
 \label{tab:regression_scheme_dependence}
 \end{center}
\end{table}

Measured main detector asymmetries depends on choice of regression schemes and are summarized in Table~\ref{tab:regression_scheme_dependence}. The regression scheme dependent uncertainty is defined as the largest difference between all of the schemes and estimated to be 0.004~ppm for horizontal transverse and 0.006~ppm for vertical transverse data set.

%\begin{center}
%\framebox[\frameboxsize][c]{Systematic error due to regression schemes dependence is $\sim$ 0.0046 ppm.}
%\end{center}

%\newpage
\subsection{Regression Time Dependence}
\label{Regression Time Dependence}

The standard regression algorithm works with 5 minute runlet averaged quantities. The detector sensitivities are averaged over each runlet and corresponding differences are used to correct for the false asymmetry for each quartet in the runlet. There is another systematic uncertainty associated with regression time period that is considered.
%The average sensitivities for a beam parameter with its corresponding differences for a runlet was used to obtain the correction for the beam parameter of a runlet. The regressed asymmetries were then extracted for each runlet using total correction. 
The effect of using slug, few hours, as time period for the regression instead of runlets was determined. 
The error weighted average sensitivities for a slug were calculated and average beam parameter differences for that slug were used to get the corrections, as shown in Equation~\ref{equ:eqCorrection2}. These slug averaged corrections were then used to regress asymmetries (Equation~\ref{equ:eqCorrection1}). 
%Run vs Runlet Correction: 
%The effect of using runlets instead of slug for regression on the extracted transverse spin asymmetry
%was determined by first calculating the error weighted average sensitivities from the elastic
%LH$_{2}$ data set with horizontal transverse polarization and then by using the average beam parameter differences
%on that period to calculate the corrections by hand and by applying them to the unregressed
%asymmetries.

\begin{equation} \label{equ:eqCorrection1}
\langle A_{reg}\rangle_{slug} = \langle A_{UnReg}\rangle_{slug} - \langle C\rangle_{slug}
\end{equation}

\begin{equation} \label{equ:eqCorrection2}
\langle C\rangle_{slug} = \sum^{6}_{i=1} \left\langle \frac{\partial A }{\partial T_{i}}\right\rangle_{slug} \langle\Delta T_{i}\rangle_{slug}
\end{equation}

where $T_{i}$'s are $X$, $X^{\prime}$, $Y$, $X^{\prime}$, $A_{E}$, and  $A_{Q}$. The slug averaged sensitivities and beam parameter differences for the data set are shown in Figure~\ref{fig:MD_v_transverse_5+1_Sensitivities} (also Figure~\ref{fig:MD_h_transverse_5+1_Sensitivities} for horizontal transverse) and Table~\ref{tab:differences} respectively.
%The impact on regression for using a different time period for averaging sensitivities and differences for horizontal and vertical transverse data set are 0.006~ppm and 0.008~ppm respectively and assigned as regression time dependence systematic uncertainties. More details in APPENDIX-\ref{Beam Normal Single Spin Asymmetry in Inelastic e-p Scattering} section~\ref{Regression Time Dependence 2}.
The impact on regressed asymmetries due to change in the regression averaging time period for horizontal and vertical transverse data set are 0.006~ppm and 0.008~ppm respectively and are assigned as regression time dependence systematic uncertainties. More details in APPENDIX-\ref{Beam Normal Single Spin Asymmetry in Inelastic e-p Scattering} section~\ref{Regression Time Dependence 2}.

%\begin{table}[!h]
%\begin{center}
%  \begin{tabular}{ c  c  c  c }
%    \hline
%    Asymmetries 		&	Barsum		&	PMTavg	&	Difference	\\
%    			 		&	[ppm]		&	[ppm]	&	[ppm]	\\
%	\hline
%	A$_{M}^{H}$ & 5.34291 &	5.34293 &	0.00002 \\
%	A$_{M}^{V}$ & 4.52568 &	4.52522 &	0.00046 \\
%    \hline
%  	\end{tabular}
%  	\caption[Barsum and PMTavg asymmetries.]{Barsum and PMTavg asymmetries.}
%  \label{tab:BarsumPMTavgAsymmetries}
%\end{center}
%\end{table}
%
%\begin{center}
%\framebox[\frameboxsize][c]{Run vs Runlet correction is $\sim$0.0066~ppm.}
%\end{center}

\begin{figure}[!h]
	\begin{center}
	\includegraphics[width=15.0cm]{figures/transverseN2DeltaChargeSensitivity}
	\end{center}
	\caption
	[Charge sensitivity for horizontal and vertical transverse polarization data set.]
	{Charge sensitivity for horizontal (top) and vertical (bottom) transverse polarization data set. Average charge sensitivities of the measured detector asymmetries extracted from the six parameter (five parameter + charge) regression at beam current 180~$\mu$A. Purple (Black) represents the charge sensitivity of the IHWP IN (OUT) data which are consistent with each other. The sensitivities of the eight \v{C}erenkov detectors vary from -0.5\% to - 2.0\% and are stable within the running period. Average non linearity is -1\% for both the cases.}
	\label{fig:transverseN2DeltaChargeSensitivity}
\end{figure}

\subsection{Nonlinearity}
\label{Nonlinearity}
The \v{C}erenkov detector signals are normalized to the charge and the charge asymmetry is actively driven to zero using a charge feedback system. The nonlinearity of the BCM electronics, the main detector electronics and target density fluctuations can induce nonlinear distortions in the charge asymmetry and hence in the measured asymmetry~\cite{mack_BCMLinearity}. This nonlinearity of the system is seen to be non-zero from the non-zero constant term in the (5+1) regressed detector asymmetries, as shown in Figure~\ref{fig:transverseN2DeltaChargeSensitivity}. For both horizontal and vertical polarization data sets, nonlinearity is found to be -1\%. At present, no proper method of handling the measured asymmetry distortion due to nonlinearity is available. The charge sensitivity of the main detector asymmetries is used as an indicator of the nonlinearity of the system and its contribution is treated as a systematic uncertainty. The nonlinearity term is multiplied with the measured asymmetry to calculate the false asymmetry~\cite{mack_nonlinearity}. The systematic uncertainties due to nonlinearity for horizontal and vertical transverse measurements are given by 0.053~ppm and 0.045~ppm respectively.


\begin{figure}[!h]
	\begin{center}
	\includegraphics[width=15.0cm]{figures/cutDependence_LH2}
	\end{center}
	\caption
	[Cut dependence study.]	
	{Cut dependence study. Shift in the central value of the regressed asymmetry for different cut widths for LH$_{2}$. The expected statistical shift is shown by the shaded region using the total number of quartets lost when a cut is applied to all parameters.}
	\label{fig:cutDependence_LH2}
\end{figure}

\subsection{Cut Dependence}
\label{Cut Dependence}
The goal of the cut dependence analysis was to assign a systematic uncertainty that comes from shifts in the mean value of the regressed asymmetry beyond statistical fluctuations after applied cuts. If linear regression is working properly, large false asymmetries in runlets with large HCBAs should be removed from the measured asymmetry after linear regression is applied and there should not be any shift in the mean value of the regressed asymmetry beyond statistical shifts (as shown in Figure~\ref{fig:cutDependence_LH2}).
The point-to-point uncertainty in going from cut $i$ to cut $j$ is estimated to be

\begin{equation} \label{equ:eqCutDependence1}
\Delta_{i \rightarrow j}^{pt-to-pt} = \left( \frac{\sigma_{j}}{\sqrt{N_{j}}} - \frac{\sigma_{i}}{\sqrt{N_{i}}} \right)
\end{equation}

%Cuts on the helicity-correlated beam parameters were used to assign a systematic error that comes from shifts in the mean value of the regressed (5+1) asymmetry after cuts are applied.

Here the $\sigma$ is the root mean square (RMS) of each HCBA.
Inclusive cuts of 7, 6, 5, 4, 3, 2.5 and 2~$\sigma$ are applied to all HCBAs and difference between regressed asymmetry with cut and without cuts are shown in Figure~\ref{fig:cutDependence_LH2}.
The observed shift in the measured asymmetry from these cuts are larger than the expected statistical shift and 2.5~$\sigma$ cuts on the HCBAs were used to assign a systematic uncertainty. The total percentage of quartets lost for cuts with respect to no cut are used to estimate the expected statistical shift, shown as the shaded region in Figure~\ref{fig:cutDependence_LH2}. Beyond a cut of 2.5~$\sigma$, most of the data were removed to extract a meaningful asymmetry. This analysis was performed to assign systematic uncertainty only, no data was removed from main data set. 
Cut dependence for horizontal and vertical transverse data set are found to be $\sim$0.064~ppm and $\sim$0.068~ppm respectively.

%\begin{center}
%\framebox[\frameboxsize][c]{Cut dependence is $\sim$0.0654~ppm.}
%\end{center}

\subsection{Fit Scheme Dependence}
\label{Fit Scheme Dependence}
A sinusoidal fit to main detector octant asymmetries is used to extract measured transverse asymmetry. So it was important to find the impact of the function on fitted asymmetry.
The measured asymmetry was fitted using four different functions, and the solutions are summarized in Table~\ref{tab:FitSchemeDependence}. 
%The phase of the fit can also affect extracted asymmetry. In this analysis two methods were used, first by fitting the phase as a parameter and  second without fitting the phase. 
%Different fit functions and corresponding measured asymmetries are shown Table~\ref{tab:FitSchemeDependence}. 
The difference in measured asymmetry obtained using standard function $A_{M}\sin(\phi+\phi_{0})+C$ and rest gives an idea about the fit function dependence of the measured asymmetry. More insightfully, the constant term in the fit function can be thought of as the parity violating asymmetry contribution to the parity conserving transverse asymmetry. The size of transverse asymmetry is much larger than the parity violating asymmetry to have any significant effect on the transverse measurement. So this PV asymmetry is buried under the fit scheme dependence and give rise to the systematic uncertainties of 0.040~ppm for horizontal and 0.083~ppm for vertical transverse data sets. 

\begin{table}[!h]
\begin{center}
  	\caption
	[Fit scheme dependence of the measured asymmetry.]
  	{Fit scheme dependence of the measured asymmetry. The fit function was varied to observe the effect on measured regressed asymmetry. The difference in asymmetry between case 1 and rest are shown.}
  \begin{tabular}{ c | c  c  c  | c  c  c }
%    \hline
    \noalign{\hrule height 1pt}
 	& \multicolumn{3}{c|}{Horizontal transverse}  & \multicolumn{3}{c}{Vertical transverse} \\ 
% 	\hline
	\cline{2-7}
    & \multirow{3}{*}{Fit Function} 	& \multirow{2}{*}{$A_{M}^{H}$}	&	Difference 	& \multirow{3}{*}{Fit Function} &	\multirow{2}{*}{$A_{M}^{V}$}	&	Difference \\
    	&		 		&					&	(1-i)			& &				&	(1-i) \\
    	&		 		&	[ppm]				&	[ppm]			& &	[ppm]				&	[ppm] \\
%    	 \hline
    \noalign{\hrule height 1pt}
	1 & $A_{M}^{H}\sin(\phi+\phi_{0}^{H})+C^{H}$	& 5.343 &0.000 & $A_{M}^{V}\cos(\phi+\phi_{0}^{V})+C^{V}$	& 4.525 &	 0.000 \\
	2 & $A_{M}^{H}\sin(\phi+\phi_{0}^{H})$ & 5.344 & 0.001 & $A_{M}^{V}\cos(\phi+\phi_{0}^{V})$ & 4.510 &	 0.015 \\
	3 & $A_{M}^{H}\sin(\phi)+C^{H}$ & 5.303 & 0.040 & $A_{M}^{V}\cos(\phi)+C^{V}$ & 4.458 & 0.067 \\
	4 & $A_{M}^{H}\sin(\phi)$ & 5.304 & 0.039 & $A_{M}^{V}\cos(\phi)$ & 4.442 &	 0.083 \\	
%    \hline
    \noalign{\hrule height 1pt}
  	\end{tabular}
  \label{tab:FitSchemeDependence}
\end{center}
\end{table}
%The measured asymmetry has negligible phase dependence. 
%
%\begin{center}
%\framebox[\frameboxsize][c]{Fit scheme dependence is $\sim$0.0524~ppm.}
%\end{center}

\begin{figure}[!h]
	\begin{center}
	\includegraphics[width=15.0cm]{figures/errorChart}
	\end{center}
	\caption
	[Summary of uncertainties on measured asymmetry for horizontal and vertical data set.]
	{Summary of uncertainties on measured asymmetry for horizontal and vertical data set. The relative total uncertainty is dominated by statistical uncertainty compared to systematic uncertainties.}
	\label{fig:errorChart}
\end{figure}

\subsection{Summary of Systematic Uncertainties}
\label{Summary of Systematic Uncertainties}
Summary of systematic uncertainties of the measured inelastic beam normal single spin asymmetry is given in Table~\ref{tab:systematic_error}. The systematic studies contain uncertainties related to the extraction of the measured asymmetry such as regression, nonlinearity, cut dependence, and detector acceptance correction. The systematic studies for horizontal and vertical transverse polarization data set were performed separately; these are summarized in Figure~\ref{fig:errorChart}. Statistical uncertainty weighted average of the systematic uncertainties from horizontal and vertical transverse data sets are considered for the total systematic uncertainty. Total uncertainty is the quadrature sum of the statistical and systematic uncertainties. The relative total uncertainty in measured asymmetry is dominated by 9\% statistical uncertainty compared to 1\% systematic uncertainty.

%\begin{equation} \label{equ:measuredAsymmetry}
%A_{M}^{in} = 5.127~\pm~0.444~\text{(stat)}~\pm~0.100~\text{(sys) ppm}.
%\end{equation}

\begin{table}[!h]
\begin{center}
  	\caption
  	[Summary of uncertainties on measured asymmetry for combined horizontal and vertical data sets.]
  	{Summary of uncertainties on measured asymmetry for combined horizontal and vertical data sets. The relative uncertainties are also shown in the table.}
  \begin{tabular}{ c | c | c }
%    \hline
    \noalign{\hrule height 1pt}
    \multirow{2}{*}{Uncertainty from}&	Contribution to $A_{M}$	&	Relative Contribution 	\\
									&	[ppm]	&	[\%] \\ 
%	\hline
    \noalign{\hrule height 1pt}
 	Statistics   					&	0.444	&	8.7	\\ 
% 	\hline
    \noalign{\hrule height 1pt}
	Regression scheme dependence  	&	0.005	& 	0.1	\\
	Regression time dependence		&	0.007	&	0.1	\\
	Non-linearity					&	0.051	& 	1.0	\\
	Cut dependence					&	0.065	& 	1.3	\\
	Fit scheme dependence			&	0.052	& 	1.0	\\ 
	Detector acceptance correction	&	0.016	& 	0.3	\\ 
	\hline
%    \noalign{\hrule height 1pt}
	Systematic only					&	0.100	& 	2.0	\\ 
%	\hline
    \noalign{\hrule height 1pt}
	Total 							& 	0.455 	& 	8.9	\\    	
%    \hline
    \noalign{\hrule height 1pt}
  	\end{tabular}
  \label{tab:systematic_error}
\end{center}
\end{table}

%\begin{table}[!h]
%\begin{center}
%  \begin{tabular}{ c  c  c }
%    \hline
%    Error from	&	Contribution to $A_{M}$		&	Section	\\
%														&	[ppm]	&	\\\hline
%	Regression scheme dependence  &	0.005	& 	\ref{Regression Scheme Dependence}	\\
%	Regression time dependence		&	0.007	&	\ref{Regression Time Dependence} \\
%	Cut dependence							&	0.065	& \ref{Cut Dependence} \\
%	Non-linearity								&	0.051	& \ref{Cut Dependence} \\
%	Fit scheme dependence				&	0.052	& \ref{Fit Scheme Dependence} \\\hline
%
%	Total 											& 	0.127 	& \\    	
%    \hline
%  	\end{tabular}
%  	\caption[Systematic error table.]{Systematic error table.}
%  \label{tab:systematic_error}
%\end{center}
%\end{table}


%\newpage
%%%%%%%%%%%%%%%%%%%%%%%%%%%%%%%%%%%%%%%%%%%%%%%%%%%%%%%%%%%%%
\section{Extraction of Physics Asymmetry}
\label{Extraction of Physics Asymmetry}
%
%\begin{landscape}
%
%%\begin{flushleft}
%\begin{table}[h]
%\begin{center}
%  \begin{tabular}{ c | c  c  c | c  c | c  c | c }
%    \hline
%    Quantity 		&	&	Asymmetry	&	& 	& Dilution & Correction & $c_{i}=A_{bi}f_{bi}$ & Reference		\\
%	\hline
%	Aluminum asymmetry & $A_{b1}$ & 8.431 $\pm$ 0.985 & ppm & $f_{b1}$ & 0.033 $\pm$ 0.002 & $c_{b1}$ & 0.033 $\pm$ 0.002 &	\href{https://qweak.jlab.org/elog/Analysis+&+Simulation/451}{ELOG 451 (Analysis)}\cite{website:elog_adesh} \\
%
%	QTOR transport channel & $A_{b2}$ & 0.000 $\pm$ 0.000 & ppm	& $f_{b2}$ & 0.033 $\pm$ 0.002 & $c_{b1}$ & 0.033 $\pm$ 0.002 & \href{https://qweak.jlab.org/doc-private/ShowDocument?docid=1655}{DocDB 1655} \cite{buddhini_transverse_technote}	\\
%
%	Beamline background & $A_{b3}$ & 0.000 $\pm$ 0.000 & ppm	& $f_{b3}$ & 0.033 $\pm$ 0.002 & $c_{b1}$ & 0.033 $\pm$ 0.002 & 	\href{https://qweak.jlab.org/doc-private/ShowDocument?docid=1655}{DocDB 1655} \cite{buddhini_transverse_technote}	\\
%	
%	Elastic asymmetry & $A_{b4}$ & -5.305 $\pm$ 0.166 & ppm	& $f_{b4}$ & 0.033 $\pm$ 0.002 & $c_{b1}$ & 0.033 $\pm$ 0.002 &	\href{https://qweak.jlab.org/doc-private/ShowDocument?docid=1655}{DocDB 1655} \cite{buddhini_transverse_technote}	\\
%
%    \hline
%  	\end{tabular}
%  	\caption[Background correction table.]{Background correction table.}
%  \label{tab:PhysicsAsymInput}
%\end{center}
%\end{table}
%%\end{flushleft}
%
%
%
%\begin{table}[h]
%\begin{center}
%  \begin{tabular}{ c  c  c  c  c  c  c }
%    \hline
%    Quantity 		&	&	Value	&	&	& &  Reference		\\
%	\hline
%	Inelastic measured asymmetry & $A^{in}_{M}$ &	4.789 $\pm$ 0.844 & ppm & & & This analysis\\
%
%	Polarization & P & 	0.879 $\pm$ 0.018 & &	& & \href{https://qweak.jlab.org/doc-private/ShowDocument?docid=1655}{DocDB 1655} \cite{buddhini_transverse_technote}	\\
%
%	Aluminum asymmetry & $A_{b1}$ & 8.431 $\pm$ 0.985 & ppm & $f_{b1}$ & 0.033 $\pm$ 0.002 &	This analysis\\
%	
%%	Aluminum dilution factor & $f_{b1}$ & 0.033 $\pm$ 0.002 & &	\href{https://qweak.jlab.org/elog/Analysis+&+Simulation/451}{ELOG 451 (Analysis)}\cite{website:elog_adesh}	\\
%
%%	Aluminum asymmetry radiative correction & $R^{A}_{b1}$ & 1.000 $\pm$ 0.000 &		&	Place holder	\\
%
%	QTOR transport channel & $A_{b2}$ & 0.000 $\pm$ 0.000 & ppm	& $f_{b1}$ & 0.033 $\pm$ 0.002 & \href{https://qweak.jlab.org/doc-private/ShowDocument?docid=1655}{DocDB 1655} \cite{buddhini_transverse_technote}	\\
%	
%%	QTOR transport channel neutral dilution & $f_{b2}$ & 0.724 $\pm$ 0.004 & &	\href{https://qweak.jlab.org/elog/Analysis+&+Simulation/451}{ELOG 451 (Analysis)}\cite{website:elog_adesh}	\\
%
%	Beamline background & $A_{b3}$ & 0.000 $\pm$ 0.000 & ppm	& $f_{b1}$ & 0.033 $\pm$ 0.002 & 	\href{https://qweak.jlab.org/doc-private/ShowDocument?docid=1655}{DocDB 1655} \cite{buddhini_transverse_technote}	\\
%	
%%	Beamline background dilution & $f_{b3}$ & 0.724 $\pm$ 0.004 & &	\href{https://qweak.jlab.org/elog/Analysis+&+Simulation/451}{ELOG 451 (Analysis)}\cite{website:elog_adesh}	\\
%	
%	Elastic asymmetry & $A_{b4}$ & -5.305 $\pm$ 0.166 & ppm	& $f_{b1}$ & 0.033 $\pm$ 0.002 &	\href{https://qweak.jlab.org/doc-private/ShowDocument?docid=1655}{DocDB 1655} \cite{buddhini_transverse_technote}	\\
%	
%%	Elastic dilution factor & $f_{b4}$ & 0.724 $\pm$ 0.004 & &	\href{https://qweak.jlab.org/elog/Analysis+&+Simulation/451}{ELOG 451 (Analysis)}\cite{website:elog_adesh}	\\
%
%%	Elastic asymmetry radiative correction & $R^{A}_{b4}$ & 1.000 $\pm$ 0.000 &		&	Place holder	\\
%%
%%	EM radiative correction & $R_{RC}$ & 1.000 $\pm$ 0.000 &		&	Place holder	\\
%%
%%	Detector bias correction & $R_{Det}$ & 1.000 $\pm$ 0.000 &		&	Place holder	\\
%
%    \hline
%  	\end{tabular}
%  	\caption[Input table.]{Input table.}
%  \label{tab:PhysicsAsymInput}
%\end{center}
%\end{table}
%
%\end{landscape}
%
%%%%%%%%%%%%%%%%%%%%%%%%%%%%%%%%%%%%%%%%%%%%%%
The beam normal single spin asymmetry from inelastic e+p scattering is obtained from measured asymmetry using Equation~\ref{equ:PhysicsAsymmetryShort} by accounting for EM radiative corrections, kinematics normalization, polarization, and backgrounds.

%\begin{equation} \label{equ:PhysicsAsymmetry}
%A_{N} = R_{RC}R_{Det}R_{Q^{2}}R_{\phi} \left[ \frac{\frac{A^{in}_{M}}{P} - A_{b1}f_{b1} - A_{b2}f_{b2} - A_{b3}f_{b3} - A_{b4}f_{b4} }{1 - f_{b1} - f_{b2} - f_{b3} - f_{b4}} \right] 
%\end{equation}
\begin{equation} \label{equ:PhysicsAsymmetryShort}
A_{N} = R_{total} \left[ \frac{\left(\frac{A^{in}_{M}}{P}\right) - \sum^{4}_{i=1} A_{bi}f_{bi} }{ (1 -\sum^{4}_{i=1} f_{bi}) } \right]
\end{equation}

Here $R_{total}$ is a correction factor for the experimental bias and radiative effects, $P$ is the beam polarization, and $A_{bi}$ is $i^{th}$ background asymmetry with fraction of backgrounds in the total detector acceptance (dilution) $f_{bi}$. The systematic corrections on the physics asymmetry and the associated uncertainties are discussed in the following sections.


\subsection{Beam Polarization}
\label{Beam Polarization}
The Hall-C M{\o}ller polarimeter and the Compton polarimeter were used to measure the beam polarization for the experiment, but only the measurements from the M{\o}ller polarimeter will be used for this analysis. 
The photocathode Quantum Efficiency was steady and hence the beam polarization was stable for the period~\cite{magee_communication}. 
%The assumptions was that, the beam polarization in the beam within few hours remain similar. 
The M{\o}ller polarimeter is only sensitive to longitudinally polarized beam. So measurements performed with the longitudinally polarized beam right after the transverse data taking was used to determine the beam polarization. 
%The M{\o}ller measurements performed with the longitudinally polarized beam right after the the transverse data taking was used to determine the beam polarization. 
The M{\o}ller runs used for this analysis are 1593 - 1599, carried out on 20th February 2012. Each run is $\sim$10 min long. Slug averaged polarizations from this M{\o}ller measurement are shown in Table~\ref{tab:polarization}. The measured beam polarization is given by $P$ = 87.50 $\pm$ 0.28 (stat) $\pm$ 0.74 (sys)\%~\cite{elog:nur_ancillary91}. Details of systematic studies for the M{\o}ller polarization measurement can be found in Q-weak internal technical document~\cite{magee_moller}.
%~\cite{magee_communication}

\begin{table}[h]
\begin{center}
  	\caption
	[Beam polarization using M{\o}ller polarimeter for Run 2 transverse data set.]  	
  	{Beam polarization using M{\o}ller polarimeter for Run 2 transverse data set~\cite{magee_moller}.}
  \begin{tabular}{ c | c | c }
%    \hline
    \noalign{\hrule height 1pt}
    \multirow{2}{*}{IHWP} &	Polarization		&	Statistical Uncertainty	\\
    			&	[\%]					&	[\%]		\\ 
%    	\hline
    \noalign{\hrule height 1pt}
	Out 		& 87.029 				& 0.398	\\
	In 		& - 87.939 			& 0.387	\\ 
%	\hline
    \noalign{\hrule height 1pt}
	Total 	& 87.497 				& 0.277	\\ 
%	\hline
    \noalign{\hrule height 1pt}
  	\end{tabular}
  \label{tab:polarization}
\end{center}
\end{table}


%M{\o}ller polarimeter is only sensitive to longitudinal polarization.
%To determine the beam polarization in the transversely polarized beam, the experiment relied on M{\o}ller measurements done with the longitudinally polarized beam close to the transverse data taking period. Since the quantum efficiency of a laser spot on the cathode is good for up to two weeks (see Subsection 3.2.1), the polarization in the beam within one to two days of the transverse measurements are similar to the beam polarization in the transverse running period.
%By design, the Møller polarimeter is only sensitive to longitudinal polarization.
%To determine the beam polarization in the transversely polarized beam, the experiment
%relied on Møller measurements done with the longitudinally polarized beam
%close to the transverse data taking period. Since the quantum efficiency of a laser
%spot on the cathode is good for up to two weeks (see Subsection 3.2.1), the polarization
%in the beam within one to two days of the transverse measurements are similar to
%the beam polarization in the transverse running period. With this assumption, Transverse
%Run I uses the beam polarization estimated [148] for the Qweak 21 % result from
%the commissioning period (see left panel in Figure 6.15).
%Table 6.8 shows the polarization results from run2 used for the transverse analysis.

%\begin{figure}[!h]
%	\begin{center}
%	\includegraphics[width=10.0cm]{figures/beamPolarization}
%	\end{center}
%	\caption[Polarization results used for transverse dataset based on the M{\o}ller measurements performed on
%February 20th, 2012 immediately after transverse measurements. The uncertainties shown are statistical only. The data points are corrected for IHWP reversal.]{Polarization results used for transverse dataset based on the M{\o}ller measurements performed on
%February 20th, 2012 immediately after transverse measurements. The uncertainties shown are statistical only. The data points are corrected for IHWP reversal.}
%	\label{fig:beamPolarization}
%\end{figure}



\subsection{Background Corrections}
\label{Background Corrections}

The largest background source in beam normal single spin asymmetry arises from elastic radiative tail. Small background contributions also come from electrons scattering from aluminum target windows, beamline scattering, and other soft neutral scattering. The analysis of the background asymmetries and their contributions to the BNSSA is described in following sections.


\begin{figure}[!h]
	\begin{center}
	\includegraphics[width=15.0cm]{figures/asymmetry_Al}
	\end{center}
	\caption
	[Azimuthal dependence of asymmetry from the 4\% downstream aluminum target.]
	{Azimuthal dependence of asymmetry from the 4\% downstream aluminum target. The uncertainties are statistical only. The octant dependence in either polarization orientation are similar to what was observed for the LH$_{2}$-cell. The asymmetry is larger than the LH$_{2}$-cell asymmetry. The fit functions used for horizontal and vertical transverse data points are $A_{M}^{H}\sin(\phi+\phi_{0}^{H}) + C^{H}$ and $A_{M}^{V}\cos(\phi+\phi_{0}^{V}) + C^{V}$ respectively.}
	\label{fig:asymmetry_Al}
\end{figure}

\subsubsection{Target Aluminum Windows}
\label{Target Aluminum Windows}
One of the important background contributions to the measured asymmetry comes from electrons scattering from the aluminum alloy target windows. 
Data were taken on the 4\% downstream aluminum alloy target to determine the size of this asymmetry.
%To determine the size of this asymmetry, we took data on the 4\% Downstream aluminum target. 
The measured regressed asymmetry for horizontal and vertical transverse are $A^{H}_{DSAl}$ = 7.892~$\pm$~1.186~ppm and $A^{V}_{DSAl}$ = 9.631~$\pm$1.768~ppm~\cite{elog:nur_ancillary43}, as shown in Figure~\ref{fig:asymmetry_Al}. Combined (error weighted) regressed aluminum alloy asymmetry is $A_{DSAl}$ = 8.432~$\pm$~0.985~ppm. This asymmetry is then scaled by a 0.9938 for azimuthal acceptance averaging (already discussed in section~\ref{Azimuthal Acceptance Correction}), which yields the asymmetry as 8.484$\pm$0.985~ppm. The acceptance difference between the upstream and downstream target windows need to correct before the background correction. This acceptance difference causes a 20\% relative difference between the mean $Q^{2}$ of the electrons coming from the upstream window compared to the downstream window, as shown in GEANT4 simulations~\cite{kmyers_qweak} ($Q^{2}_{USAl}$=0.8$\times Q^{2}_{DSAl}$). The beam normal single spin asymmetry from nuclei at forward angle scattering asymmetry is proportional to $\sqrt{Q^{2}}$ as described in theoretical models~\cite{PhysRevC.72.034602, PhysRevC.77.044606}. So, asymmetry for upstream aluminum target can be calculated as $A_{USAl}$ = $\sqrt{0.8} A_{DSAl}$ = 7.589~ppm. Downstream and upstream aluminum target windows are expected to contribute equally~\cite{kmyers_qweak} to the aluminum dilution in the main detector asymmetries resulting in an effective aluminum asymmetry of $A_{Al}$ =  $(A_{DSAl}+A_{DSAl})/2$ = 8.036~ppm. An additional systematic uncertainty of 0.08$\times A_{Al}$ is assigned for the system non-linearity (more details in APPENDIX~\ref{Beam Normal Single Spin Asymmetry in Inelastic e-p Scattering}). 
The polarization corrected asymmetry for background windows correction is $A_{b1}$ = $A_{Al}$/$P$ = 9.185~$\pm$~1.409~ppm.
%\begin{equation} \label{equ:AluminumAsymmetry}
%A^{in}_{PHYS} = \frac{\frac{A^{in}_{M}}{P} - A^{Al}f^{Al}}{1 - f^{Al}}
%\end{equation}
The measured aluminum windows dilution 
%, the fraction of the total rate seen by the main detectors that is from the target's aluminum windows, 
is $f_{b1}$ = 0.033~$\pm$~0.002~\cite{presentation:josh_1891}. 
Dedicated measurements were performed with different pressures of hydrogen gas in the target cell. Using the known pressure of hydrogen gas at different points, the pressure was extrapolated to zero. % Data was also taken with an evacuated target cell. 
The correction to the physics asymmetry from aluminum alloy windows is $c_{b1}$ = $\kappa PA_{b1}f_{b1}$ = 1.427~ppm, where $\kappa$ = $(R_{total}/P)/(1-f_{total})$.

\subsubsection{Beamline Scattering}
\label{Beamline Scattering}
Another correction accounts for scattering sources in the beam line ($b2$), with an asymmetry ($A_{b2}$) measured, along with its dilution ($f_{b2}$), by blocking two of the eight openings in the first of the three Pb collimators with tungsten. The measured asymmetry in the blocked octants detectors was correlated with different background detectors located outside the acceptance of the main detectors for scaling during the primary measurement, assuming a constant dilution~\cite{elog:kent_analysis782}. The variation of upstream luminosity monitor asymmetry with octant during longitudinal running can provide a good indication of the beamline scattering asymmetry. The maximum variation before and after the transverse data collection period (during longitudinal running) $\Delta A_\textrm{USLumi}$ = 3.534~$\pm$~0.16~ppm was used to estimate the beamline scattering asymmetry. A very simple postulate was considered: that measured main detector asymmetry has a background with a fixed fraction and an asymmetry that scales linearly with that measured in the background monitors and USLumis. The scale factor was measured directly, correlating the MD asymmetry to background asymmetries, and was estimated to be 0.0085~$\pm$~0.0016~\cite{elog:manolis_analysis1191} from longitudinal period. The signal drops by an order of magnitude lower for inelastic scattering compared to elastic, whereas beamline background remains similar. Hence an additional factor of 10 was multiplied to incorporate the signal drop. The beamline background does not depend on polarization and is not corrected for it. Then, asymmetry for beam line scattering is given by $A_{b2}$ = $\Delta A_\textrm{USLumi}\times 0.085$ = 0.300~$\pm$~0.058~ppm. 
The beamline scattering dilution factor for inelastic running is an order of magnitude larger than in the elastic kinematic setting. The total rate at the inelastic peak drops to 10\% of the total rate at the elastic peak, whereas the number of events originating in the beamline remains similar. The measured dilution for inelastic beamline scattering is 0.018~$\pm$~0.001~\cite{leacock_qweak, elog:mack_analysis784}. A 50\% uncertainty on the dilution was assigned to allow the sinusoidal modulation specific to the BNSSA. The beamline scattering dilution used for the background correction is $f_{b2}$ = 0.018~$\pm$~0.009. 
The correction to the physics asymmetry due to beam line scattering is $c_{b2}$ = $\kappa PA_{b2}f_{b2}$ = 0.025~ppm.

\subsubsection{Other Neutral Background}
\label{Other Neutral Background}
An additional correction was applied to include soft neutral backgrounds ($b3$) arising from secondary interactions of scattered electrons in the collimators and magnet, and was not accounted in the blocked octant studies~\cite{elog:mack_analysis714}. 
%A dedicated M{\o}ller~\cite{elog:buddhini_analysis553} asymmetry was measured with 0.877 GeV transverse electron beam as $A_{M{\o}ller}$ = 9.33~$\pm$~0.54~ppm. This measurement is used for the current analysis at 1.155~GeV to give an upper bound on the asymmetry. If the primary electron interaction at scraping is dominated by M{\o}ller scattering then the expected neutral asymmetry can be obtained as $A_{b3}$ = $A_{M} - A_{M{\o}ller}$ = 4.235~$\pm$~0.706~ppm.
The primary electron interaction at scraping is partially coming from M{\o}ller scattering, but the source of the asymmetry is not well understood. 
%The asymmetry from M{\o}ller scattering has a similar size but opposite sign compared to the measured transverse asymmetry~\cite{elog:buddhini_analysis553}.
The other neutral background asymmetry could be as large as 5 ppm (size of the transverse asymmetry).
To make the sign of the asymmetry uncertain, the asymmetry for other neutral background was assumed to be $A_{b3}$ = 0.000~$\pm$~10.000~ppm. Here, uncertainty of 100\% of the measured transverse asymmetry was assigned to give an upper bound on the neutral background asymmetry. 
The neutral background dilution for the inelastic scattering has been measured as $f_\textrm{neutral}$ = 0.0520~$\pm$~0.0040~(stat)~$\pm$~0.0014~(sys)~\cite{rakitha_neutral_background}. The dilution for the other neutral background was obtained by subtracting the blocked octant background from the total neutral background measured by the main detector and is given by $f_{b3}$ = $f_\textrm{neutral} - f_{b2}$ = 0.034~$\pm$~0.010. The correction to the physics asymmetry due to other neutral background is $c_{b3}$ = $\kappa PA_{b3}f_{b3}$ = 0.000~ppm.

%Kent Paschke request: assume QTOR channel asymmetry is dominated by Moellers rather than elastics, hence change A = -0.2 to 0.0. The asymmetry increased by 2 ppb. The systematic error changed in the last digit. The correction is made properly but in the "somewhat orthogonalized list of corrections" in the output file, the quantity f2*A2/(1-ftotal) is used and is zero since A2 = 0. The 2 ppb correction mentioned above comes thru the dilution factor in the denominator of (Amsr/P)/(1-f). Gonna have to work on the "somewhat" in that summary of corrections. 

\begin{figure}[!h]
	\begin{center}
	\includegraphics[width=15.0cm]{figures/qtorScan}
	\end{center}
	\caption
	[Simulation of contributions from elastic and inelastic e+p and elastic Al from upstream and downstream target windows.]
	{
	%Inelastic transverse data set. Vertical transverse are shown in parenthesis, rest are for horizontal transverse. Number of runs in each settings are shown on top panel. Each point corresponds to $\sim$1~hour of data. LH$_{2}$ runs are shown in blue square points, 4\% downstream aluminum (Al) alloy targets are shown in purple circles, and runs with Carbon target are shown in orange diamond points. The simulated curves are from GEANT 3 simulation~\cite{elog:adesh_analysis837}. The simulated curves for LH$_{2}$ inelastic, Al elastic, and Al quasi elastic were multiplied by a factor of 10 for better visualization.
Simulation of contributions from elastic and inelastic e+p, and elastic e+Al scattering from upstream (US) and downstream (DS) target windows~\cite{elog:adesh_analysis837}. All but elastic e+p events have been multiplied by 10 for better visualization.}
	\label{fig:qtorScan}
\end{figure}


\subsubsection{Elastic Radiative Tail}
\label{Elastic Radiative Tail}
The largest background correction comes from the elastic radiative tail ($b4$). The polarization corrected measured elastic transverse asymmetry was $A_{T}^{el}$ = -5.345~$\pm$~0.067~(stat)~$\pm$~0.076~(sys)~ppm~\cite{presentation:BPWQweakTransverse_02_21_14}. The elastic physics asymmetry from the LH$_{2}$-cell is similar in magnitude to the inelastic asymmetry but has the opposite sign.
The elastic asymmetry was measured at $Q_{el}^{2}$ = 0.0250~$\pm$~0.0006~(GeV/c)$^{2}$~\cite{PhysRevLett.111.141803} where as inelastic measurement was at $Q_{in}^{2}$ = 0.0209~$\pm$~0.0005~(GeV/c)$^{2}$ (shown in Figure~\ref{fig:inelastic_Q2}), hence it is necessary to scale it to the inelastic peak. The transverse asymmetry is proportional to $\sqrt{Q^{2}}$~\cite{PhysRevC.72.034602, PhysRevC.77.044606}. The polarization and $\sqrt{Q^{2}}$ corrected elastic asymmetry is given by $A_{b4}$ = $\sqrt{\frac{Q_{in}^{2}}{Q_{el}^{2}}}A_{T}^{el}$ = -4.885~$\pm$~0.093~ppm.
As $\sim$70\% of the total signal in the inelastic peak was from elastic radiative tail (Figure~\ref{fig:qtorScan}), it was important to tackle it carefully. 
A GEANT simulation was used to extract elastic dilution. Dedicated measurements were taken at both sides of the inelastic peak (at QTor current 6000~A and 7300~A) to verify the simulation. A $\sim$10\% discrepancy was observed between current mode data and GEANT simulated signal at the inelastic peak, as shown in Figure~\ref{fig:elasticDilutionDiscrepancy}. In order to incorporate this discrepancy, a 10\% systematic uncertainty was assigned to the elastic dilution for this preliminary analysis. A more detailed simulation is ongoing to explore this difference. The signal size for inelastic transverse is $\sim$2-3 times smaller than that of the elastic signal. Although the signal reduces for inelastic, the nonlinearity in the detector remains the same and might be responsible for this discrepancy.
The simulated elastic dilution factor is given by $f_{b4}$ = 0.701~$\pm$~0.070~\cite{elog:adesh_analysis837, elog:nur_ancillary59}. The correction to the physics asymmetry due to the elastic radiative tail is $c_{b4}$ = $\kappa PA_{b4}f_{b4}$ = -16.129~ppm.

\begin{figure}[!h]
	\begin{center}
	\includegraphics[width=15.0cm]{figures/elasticDilutionDiscrepancy}
	\end{center}
	\caption
	[The residual of yield using Data and simulation from GEANT 3.]
	{The residual of yield using Data and simulation from GEANT 3~\cite{elog:adesh_analysis837} are shown in the figure. A $\sim$10\% discrepancy was observed at inelastic peak (6700~A) between data and simulation for matching them at elastic peak (8921~A). Beamline background correction to the yield did not improve the discrepancy. }
	\label{fig:elasticDilutionDiscrepancy}
\end{figure}


\subsection{Other Corrections}
\label{Other Corrections}
Another set of corrections is used to remove all the experimental bias from the measured asymmetry before extracting BNSSA. The measured asymmetry is corrected for the electromagnetic (EM) radiative corrections, light weighting on the \v{C}erenkov detector, and $Q^{2}$ precision. These corrections are considered as independent factors and are applied to the measured asymmetry.

%Electromagnetic (EM) radiation causes energy loss and depolarization [154] of the electrons. To extract the beam normal single spin asymmetry at the effective $Q^{2}$ and beam polarization, the measured $A_{N}$ needs to be corrected for these EM radiative effects. The deduced radiative correction for elastic e+p scattering from simulations with and without bremsstrahlung, using methods described in Refs.~\cite{PhysRevLett.82.1096, PhysRevC.69.065501} was found to be $R_{RC}$ = 1.010~$\pm$~0.005. The same radiative correction was used for this dataset.
%$R_{Det}$ = 1.000~$\pm$~0.000 accounts for the measured light variation and nonuniform $Q^{2}$ distribution across the detector bars. $R_{Bin}$ = 1.000~$\pm$~0.000 is an effective kinematics correction~\cite{PhysRevC.69.065501} that corrects the asymmetry from $\langle A(Q^{2})\rangle$ to $A(\langle Q^{2}\rangle)$, and $R_{Q^{2}}$ = 1.000~$\pm$~0.030 represents the precision in calibrating the central $Q^{2}$.


\subsubsection{Radiative Correction}
\label{Radiative Correction}
The energy loss and depolarization of the electrons is a result of electromagnetic (EM) radiation~\cite{PhysRev.114.887}. The measured asymmetry needs to be corrected for these EM radiative effects to obtain the beam normal single spin asymmetry at the effective $Q^{2}$ and beam polarization. The deduced radiative correction for elastic e+p scattering from simulations with and without bremsstrahlung, using methods described in Refs.~\cite{PhysRevLett.82.1096, PhysRevC.69.065501}, was found to be $R_{RC}$ = 1.010~$\pm$~0.004~\cite{buddhini_qweak}. The same radiative correction was used for this data set as there were no existing simulations available for inelastic e+p scattering. This correction does not have a significant impact in the final asymmetry, hence it was not unreasonable to use the existing elastic simulation result.

%$R_{RC}$ = $R_{EB}\times R_{IB}\times R_{depolarization}$

\subsubsection{Detector Bias Correction}
\label{Detector Bias Correction}
The measured light variation and nonuniform $Q^{2}$ distribution across the detector bars affects the measured asymmetry and need to be accounted for in the final BNSSA extraction. 

\begin{equation} \label{equ:eqQ2Acceptance1}
R_{Det} = \frac{A_{\textrm{no-bias}}^{\textrm{sim}}}{A_{\textrm{bias}}^{\textrm{sim}}} = \sqrt{\frac{(Q^{2})_{\textrm{no-bias}}^{\textrm{sim}}}{(Q^{2})_{\textrm{bias}}^{\textrm{sim}}} }
\end{equation}

Here, $A^{\textrm{sim}}_{\textrm{bias}}$ and $A^{\textrm{sim}}_{\textrm{no-bias}}$ are the simulated asymmetries with and without light-collection bias respectively.
The detector bias correction used for this analysis is $R_{Det}$ = 0.998~$\pm$~0.001 and is obtained using elastic transverse simulation results~\cite{elog:peiqing_analysis589, buddhini_qweak}.

\begin{figure}[!h]
	\begin{center}
	\includegraphics[width=15.0cm]{figures/inelastic_Q2}
	\end{center}
	\caption
	[The $Q^{2}$ from GEANT 3 simulation.]
	{The $Q^{2}$ from GEANT 3 simulation~\cite{elog:nur_ancillary44}. The $Q^{2}$ was weighted by cross section and did not include any internal bremsstrahlung in the simulation (left panel). The simulated scattering angle is also shown in the right panel.}
	\label{fig:inelastic_Q2}
\end{figure}

\subsubsection{$Q^{2}$ Precision}
\label{Q2 Precision}

The $Q^{2}$ for inelastic e+p scattering was determined using GEANT 3 simulation and was found to be 0.0209~$\pm$~0.0005~$(GeV/c)^{2}$~\cite{elog:nur_ancillary44}, as shown in Figure~\ref{fig:inelastic_Q2}. Internal bremsstrahlung was not included in the simulation. The simulation was benchmarked by the tracking mode experimental data to represent the geometry of the experimental setup, collimation, and magnetic spectrometer.
The cross section weighted $Q^{2}$ was simulated at main detector using the reaction $e + p \rightarrow e + n + \pi^{+}$. The two-body scattering process, and energy and momentum conservation were used to do the calculation. The scattered electron energy, and $Q^{2}$ are expressed as 
%$E^{\prime} = RANDOM()\times(E_{in} - M_{e}) + M_{e}$ $Q_{2} = 4EE^{\prime}sin^{2}\theta$

\begin{equation} \label{equ:eqQ2Calculation}
\begin{split}
E^{\prime} &= RANDOM()\times(E_{in} - M_{e}) + M_{e} \\
Q^{2} &= 4EE^{\prime}sin^{2}\theta,
\end{split}
\end{equation}

\noindent 
where $E_{in}$ is the incident beam, $M_{e}$ is electron mass, and $\theta$ is scattering angle. 
A 2.0\% run to run variation of $Q^{2}$ was seen from the tracking data and added as systematic uncertainty in $Q^{2}$ estimation. It was important to propagate the precision of $Q^{2}$ in the final physics asymmetry. Based on theory~\cite{Afanasev200448}, the transverse beam spin asymmetries $A_{N}$ at low $Q^{2}$ behave like

\begin{equation} \label{equ:eqQ2Acceptance1}
A_{N} \approx \sqrt{Q^{2}} = m\sqrt{Q^{2}}.
\end{equation}

\begin{equation} \label{equ:eqQ2Acceptance2}
dA_{N} = \pm \frac{1}{2} \frac{m}{\sqrt{Q^{2}}} dQ^{2}
= \pm \frac{1}{2} \frac{34.7}{\sqrt{0.02078}} 0.0005
=  0.0601 ppm
\end{equation}

Using Equation~\ref{equ:eqQ2Acceptance1} on $Q^{2}$ and a 5~ppm measured asymmetry, the proportionality constant in the above relation can be calculated as 34.7~ppm/(GeV/c). The estimated uncertainty on the measured asymmetry due to the uncertainty in determining $Q^{2}$ is 0.061~ppm (Equation~\ref{equ:eqQ2Acceptance2}). 
%This is a 1.2\% relative correction. 
A correction of $R_{Q^{2}}$ = 1.000~$\pm$~0.012 was applied to include the precision in calibrating the central value of $Q^{2}$.

%\begin{equation} \label{equ:eqQ2Acceptance2}
%\begin{split}
%dB_{n} = \pm \frac{1}{2} \frac{m}{\sqrt{Q^{2}}} dQ^{2} \\
%= \pm \frac{1}{2} \frac{31.0}{\sqrt{0.02078}} 0.0005\\
%=  0.0538 ppm
%\end{split}
%\end{equation}


%\begin{center}
%\framebox[\frameboxsize][c]{$Q2$ Acceptance is $\sim$0.0601~ppm.}
%\end{center}


\subsection{Beam Normal Single Spin Asymmetry}
Summary of required quantities to extract the beam normal single spin asymmetry from the transverse data set presented so far using

\begin{equation} \label{equ:PhysicsAsymmetry}
A_{N} = R_{RC}R_{Det}R_{Q^{2}}R_{\phi} \left[ \frac{\left(\frac{A^{in}_{M}}{P}\right) - A_{b1}f_{b1} - A_{b2}f_{b2} - A_{b3}f_{b3} - A_{b4}f_{b4} }{1 - f_{b1} - f_{b2} - f_{b3} - f_{b4}} \right] 
\end{equation}
is shown in Table~\ref{tab:PhysicsAsymInput}. Equation~\ref{equ:PhysicsAsymmetryShort} has been expanded to obtain Equation~\ref{equ:PhysicsAsymmetry}. 
%The input quantities for Equation~\ref{equ:PhysicsAsymmetry} are summarized in the Table~\ref{tab:PhysicsAsymInput}.
Using all the input values in Equation~\ref{equ:PhysicsAsymmetry} gives the beam normal single spin asymmetry in inelastic e+p scattering

\begin{table}[!h]
\begin{center}
  	\caption
  	[Summary of input quantities to extract BNSSA.]
  	{Summary of input quantities to extract BNSSA. The measured regressed asymmetry is corrected for detector acceptance using the factor provided in the table. The table shows the contributions of normalization factors on $A_{M}^{in}$, then the properly normalized contributions from other sources. Background corrections listed here include only $R_\textrm{total}f_{i}A_{i}/(1-f_{total})$. Uncertainties in BNSSA due to dilution fraction and background asymmetry uncertainties are noted separately.
  	}
  \begin{tabular}{ l | c | c | c }
%    \hline
    \noalign{\hrule height 1pt}
    \multicolumn{4}{c}{Input parameters} \\
    \hline    
    Measured asymmetry ($A_{M}^{in}$)	&	\multicolumn{3}{l}{5.095 $\pm$ 0.455~ppm}  \\
    Beam polarization		(P)			&	\multicolumn{3}{l}{0.875 $\pm$ 0.008} \\
    Detector acceptance correction		&	\multicolumn{3}{l}{0.9938} \\
%	  \hline 
    \noalign{\hrule height 1pt}
    \multicolumn{4}{c}{Background corrections} \\
    \hline
    \multirow{3}{*}{Quantity} &	Asymmetry & Dilution & Correction \\
    &	($A_{bi}$) & ($f_{bi}$) & $c_{i}=\kappa PA_{bi}f_{bi}$ \\
    & [ppm] &  & [ppm] \\
	\hline
	Target windows (b1) 		& 9.185 $\pm$ 1.409 		& 0.033 $\pm$ 0.002 & 1.427 \\
	Beamline scattering (b2)	& 0.300 $\pm$ 0.058 		& 0.018 $\pm$ 0.009 & 0.025 \\
	Other neutral bkg. (b3) 	& 0.000 $\pm$ 10.000 	& 0.034 $\pm$ 0.010 & 0.000 \\
	Elastic asymmetry (b4) 	& -4.885 $\pm$ 0.093		& 0.701 $\pm$ 0.070 & -16.129 \\
%    \hline 
    \noalign{\hrule height 1pt}
    \multicolumn{4}{c}{Other corrections} \\
    \hline
    Radiative correction ($R_{RC}$)		&	\multicolumn{3}{l}{1.010 $\pm$ 0.004} \\
    Detector bias ($R_{Det}$)					&	\multicolumn{3}{l}{0.998 $\pm$ 0.001} \\
    $Q^{2}$ acceptance ($R_{Q^{2}}$)	&	\multicolumn{3}{l}{1.000 $\pm$ 0.012} \\
%	Azimuthal acceptance ($R_{\phi}$)	&	\multicolumn{3}{l}{1.006 $\pm$ 0.004} \\
%	\hline 
    \noalign{\hrule height 1pt}
  	\end{tabular}
  \label{tab:PhysicsAsymInput}
\end{center}
\end{table}



\begin{equation} \label{equ:FinalResult}
A_{N} = 42.27\pm2.45~\text{(stat)}\pm15.73~\text{(sys)}~\text{ppm}
\end{equation}
for the effective kinematics of acceptance averaged electron energy $\langle E\rangle$ = 1.155~$\pm$~0.003~GeV, $\langle Q^{2}\rangle$ = 0.0209~$\pm$~0.0005~(GeV/c)$^{2}$ and an average scattering angle $\langle\theta\rangle$ = 8.3~$\pm$~1.3\degrees{}. The contributions from the different uncertainty sources into the final measurement are summarized in Figure~\ref{fig:physicsErrorChart}. The dominant correction to the asymmetry comes from the elastic dilution tail whereas the dominant uncertainty on the measured asymmetry comes from statistics.

\begin{figure}[!h]
	\begin{center}
	\includegraphics[width=15.0cm]{figures/physicsErrorChart}
	\end{center}
	\caption
	[Summary of uncertainties in inelastic beam normal single spin asymmetry extraction.]
	{Summary of uncertainties in inelastic beam normal single spin asymmetry extraction. Measurement systematic contains the systematic uncertainties related to the extraction of the physics asymmetry such as regression, nonlinearity and acceptance averaging. The uncertainties are in ppm and the corresponding relative uncertainties are shown in parentheses.}
	\label{fig:physicsErrorChart}
\end{figure}


%\begin{table}[!h]
%\begin{center}
%  	\caption
%  	[Systematic error table.]
%  	{Systematic error table.}
%  \begin{tabular}{ c | c | c }
%%    \hline
%    \noalign{\hrule height 1pt}
%    \multirow{2}{*}{Uncertainty from}	&	Uncertainty	&	Relative uncertainty \\
%							&	[ppm]	&	[\%] \\
%%	\hline
%    \noalign{\hrule height 1pt}
%	$A_{M}^{in}$		&	2.36		&	5.9 \\
%	P				&	0.26		&	0.7 \\
%	$A_{b1}$			&	0.17		&	0.4 \\
%	$A_{b2}$			&	0.55		&	1.4 \\
%	$A_{b3}$			&	0.02		&	0.1 \\
%	$A_{b4}$			&	0.47		&	1.2 \\
%	$f_{b1}$			&	0.30		&	0.7 \\
%	$f_{b2}$			&	0.13		&	0.3 \\
%	$f_{b3}$			&	1.70		&	4.3 \\
%	$f_{b4}$			&	13.96	&	35.2 \\
%	$R_{RC}$			&	0.00		&	0.0 \\
%	$R_{Det}$			&	0.00		&	0.0 \\
%	$R_{Bin}$			&	0.00		&	0.0 \\
%	$R_{Q^{2}}$		&	1.19		&	3.0 \\
%%	\hline
%    \noalign{\hrule height 1pt}
%	Total					& 	14.33 	&	36.1 \\
%%    \hline
%    \noalign{\hrule height 1pt}
%  	\end{tabular}
%  \label{tab:PhysicsError}
%\end{center}
%\end{table}

%\begin{center}
%\framebox[\frameboxsize][c]{Extracted physics asymmetry $A_{N}$ = 39.71~$\pm$2.36 (stat)~$\pm$~14.33 (sys)~ppm.}
%\end{center}



%%%%%%%%%%%%%%%%%%%%%%%%%%%%%%%%%%%%%%%%%%%%%%%%%%%%%%%%%%%%%
\section{Comparison With Model Calculation}
\label{Comparison With Model Calculation}

No existing model calculation for beam normal single spin asymmetry was available at Q-weak kinematics during this analysis. Pasquini et al.~\cite{presentation:pasquini_Mainz} presented beam asymmetry in inelastic electron scattering (as shown in Figure~\ref{fig:InrtoChapter}) for large scattering angle at energies E = 0.424, 0.570, 0.855 GeV. The BNSSA were calculated separately for $\Delta$ and N intermediate states. Total asymmetry was the sum of these two intermediate states. 
Large asymmetries were observed in the forward region; these are dominated by quasi Virtual Compton Scattering (VCS) kinematics where one exchanged photon becomes quasi-real.
%Large asymmetries in the forward region dominated by quasi-VCS kinematics where one exchanged photon becomes quasi-real
These asymmetries are  sensitive to $\gamma^{\ast}\Delta\Delta$ form factors and can be a unique tool to study it~\cite{Alexandrou2009115}. 

%large asymmetries in the forward region dominated by quasi-VCS kinematics where one exchanged photon becomes quasi-real~\cite{presentation:pasquini_Mainz}
%
%* large asymmetries in the forward region
%* sensitive to $\gamma^{\ast}\Delta\Delta$ form factors~\cite{Alexandrou2009115}

\begin{figure}[!h]
	\begin{center}
	\includegraphics[width=15.0cm]{figures/theoryAsymmetryPasquini}
	\end{center}
	\caption
	[BNSSA asymmetry calculation from Pasquini et al.]
	{BNSSA asymmetry calculation from Pasquini et al. The points are taken from~\cite{presentation:pasquini_Mainz}. Then, the calculation is fitted with a function of the form $f(\theta_\textrm{lab}) = exp(p_{0}+p_{1}\theta_\textrm{lab})$ and interpolated to Q-weak $\theta_\textrm{lab}$ value. }
	\label{fig:theoryAsymmetryPasquini}
\end{figure}

These asymmetries were interpolated to forward angle up to $\theta_\textrm{lab}\textless$5\degrees{} using a suitable fit for all available three energies from~\cite{presentation:pasquini_Mainz}, as shown in Figure~\ref{fig:theoryAsymmetryPasquini}. The asymmetries were obtained at $\theta_\textrm{lab}$ = 8.35\degrees{} for three energies and extrapolated to Q-weak energy E = 1.155 GeV in Figure~\ref{fig:theoryAsymmetryPasquiniEnergyDependence}. Using this hand waving toy model, the obtained BNSSA is $A_{N}[\textrm{model}]$ = 12.15~ppm at Q-weak kinematics. The asymmetry from this analysis, $A_{N}[\textrm{Q-weak}]$ = 42.27~$\pm$~15.92~ppm is also shown in the Figure~\ref{fig:theoryAsymmetryPasquiniEnergyDependence}.

\begin{figure}[!h]
	\begin{center}
	\includegraphics[width=15.0cm]{figures/theoryAsymmetryPasquiniEnergyDependence}
	\end{center}
	\caption
	[BNSSA asymmetry calculation from Pasquini et al. and its extension.]
	{BNSSA asymmetry calculation from Pasquini et al. and its extension. The asymmetries from Figure~\ref{fig:theoryAsymmetryPasquini} at $\theta_\textrm{lab}$ = 8.35\degrees{} are plotted here. A fit function of the form $f(E) = exp(p_{0}+p_{1}E)$ is used to extrapolate the asymmetry to the desired Q-weak kinematic region ($E$ = 1.155 GeV).}
	\label{fig:theoryAsymmetryPasquiniEnergyDependence}
\end{figure}


%%%%%%%%%%%%%%%%%%%%%%%%%%%%%%%%%%%%%%%%%%%%%%%%%%%%%%%%%%%%%%
%\section{BNSSA Leakage on Parity Violating Asymmetry}
%\label{BNSSA Leakage on Parity Violating Asymmetry}
%
%The BNSSA in electron-proton scattering can create false asymmetry into the parity violating asymmetry measurements when the azimuthal symmetry of the detector array is broken and for residual transverse polarization in the electron beam.
%
%The transverse measurement discussed in this chapter was used to determine to assign a systematic uncertainty for the parity violating asymmetry to take into account the false asymmetry generated by BNSSA.


%%%%%%%%%%%%%%%%%%%%%%%%%%%%%%%%%%%%%%%%%%%%%%%%%%%%%%%%%%%%%%
\section{BNSSA in Nuclear Targets}
\label{BNSSA in Nuclear Targets}

In this chapter, the inelastic beam normal single spin asymmetry measurements in e-p scattering have been discussed. In addition to the inelastic data from the proton, Q-weak has data on the beam normal single spin asymmetry measurements from several other physics processes. Few of these measurements are the first of their kind and carry interesting physics. The measured regressed (5+1) asymmetries on liquid hydrogen cell, 4\% thick downstream aluminum alloy, and a 1.6\% thick downstream carbon foil are summarized in Table~\ref{tab:transverse_inelastic_asymmetry_nuclear_target}. The relative statistical precision of the measurements are also shown in the Table. The analysis of these data is ongoing and expected to test model calculations of beam normal single spin asymmetry.


\renewcommand{\arraystretch}{1.5} % make cell wider
\begin{table}[!h]
 \begin{center}
   \caption
	[Measured regressed (5+1) asymmetries in inelastic electron-nucleon scattering for transverse polarized beam.]
   {Measured regressed (5+1) asymmetries in inelastic electron-nucleon scattering for transverse polarized beam. Horizontal and vertical transverse data set are shown separately. The combined (error weighted average) asymmetries are also noted. The inelastic peak is at QTor current 6700~A. The other QTor currents were taken to improve the simulation for elastic radiative tail.}
  \begin{tabular}{ c | c | c  c  c | c  c }
%    \hline
    \noalign{\hrule height 1pt}
    \multirow{4}{*}{Pol.} & \multicolumn{6}{c}{Asymmetry [ppm]} \\ 
   	\cline{2-7}
    & \multicolumn{6}{c}{QTor currents} \\ 
   	\cline{2-7}
%	\hline
	& 6000 A & & 6700 A & &  \multicolumn{2}{c}{7300 A}\\
	\cline{2-7}%
%	\hline
	& LH$_{2}$ & LH$^{\dagger}_{2}$ & Al$^{\dagger\dagger}$ & $^{12}$C &  LH$_{2}$ & Al \\
%	\cline{2-7}%\hline
    \noalign{\hrule height 1pt}
%    \hline
%	Hor. & \pbox{2cm}{7.212$\pm$0.688} & \pbox{2cm}{(4.525$\pm$0.806)\\ 5.343$\pm$0.532} & \pbox{2cm}{(9.631$\pm$1.768)\\ 7.892$\pm$1.186} & \pbox{2cm}{10.190$\pm$1.863} & \pbox{1.8cm}{0.967$\pm$0.477} & \pbox[c][2cm][c]{2cm}{-1.245$\pm$1.087} \\ 
	Hor. & 7.212$\pm$0.688 & 5.343$\pm$0.532 & 7.892$\pm$1.186 & 10.190$\pm$1.863 & 0.967$\pm$0.477 & -1.245$\pm$1.087 \\
	Ver. &  & 4.525$\pm$0.806 & 9.631$\pm$1.768 & & & \\ 
    \hline
	\multirow{2}{*}{Com.} & 7.212$\pm$0.688 & 5.095$\pm$0.444 & 8.432$\pm$0.985 & 10.190$\pm$1.863 & 0.967$\pm$0.477 & -1.245$\pm$1.087 \\
	 & (9.5\%) & (8.7\%) & (11.7\%) & (18.3\%) & (49.3\%) & (87.3\%) \\
%    \noalign{\hrule height 1pt}
%	   Beam current $I$ [$\mu$A] & 180 & 180 & 60 & 75 & 180 & 60 \\
    \noalign{\hrule height 1pt}
   \end{tabular}
 \label{tab:transverse_inelastic_asymmetry_nuclear_target}
 \end{center}
\end{table}
\renewcommand{\arraystretch}{1.0} % make cell wider


%%%%%%%%%%%%%%%%%%%%%%%%%%%%%%%%%%%%%%%%%%%%%%%%%%%%%%%%%%%%%
\section{Conclusion}
\label{Conclusion}

The Q-weak collaboration has made a 35\% relative measurement of the beam normal single spin asymmetry of $A_{N} = 42.27\pm2.45~\text{(stat)}\pm15.73~\text{(sys)}~\text{ppm}$ using transversely polarized 1.155~GeV electrons scattering in-elastically from protons with a $Q^{2}$ of 0.0209 (GeV/c)$^{2}$. This is the first measurement of the beam normal single spin asymmetry in inelastic e-p scattering. This measurement would be an excellent test of theoretical calculations. Unfortunately, at the time of this analysis, there was no existing theoretical calculation or model to compare with the data. Hopefully this thesis will encourage theoreticians to produce new calculations. 


%Going further, Qweak beam normal single spin asymmetry measurements can be used to estimate the real part of the two-photon exchange with the use of dispersion relations. This will provide a valuable cross-check of both the dispersion relations and the models of the real part of the two-photon exchange process.