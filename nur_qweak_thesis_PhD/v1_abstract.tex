%\abstract
%\vspace{3.5mm}

%The Standard Model has been successful in describing most elementary particle data although the model is known to be incomplete. The objective of the Q-weak experiment is to challenge the predictions of the Standard Model in low $Q^{2}$ range and search for new physics at the TeV scale through a 4\% measurement of the weak charge of proton via the parity-violating asymmetry ($\sim$250 ppb) in elastic electron-proton scattering. There is a parity conserving beam normal single spin asymmetry (BNSSA), or transverse asymmetry, $B_{n}$ on LH$_{2}$ with a sin($\phi$)-like dependence due to two-photon exchange. $B_{n}$ from electron-nucleon scattering is also a unique tool to study the $\gamma^{*}\Delta\Delta$ form factors. The magnitude of $B_{n}$ is few ppm which is an order larger than our expected PV asymmetry for Q-weak. 

%We need to measure inelastic dilution at the elastic peak, even if most of it comes from Aluminum, their contribution to the parity violating inelastic asymmetry will be non-negligible. We wanted to study it in detail because the combination of few percent residual transverse polarization in the beam in addition to the potentially small broken azimuthal asymmetries in our detector might lead to few ppb corrections to the Q-weak data. Also $B_{n}$ provides direct access to the imaginary part of the two-photon exchange amplitude. It will be interesting to see the magnitude of $B_{n}$ in the N -to-$\Delta$ region which has never been measured before.

The Q-weak experiment in Hall-C at the Thomas Jefferson National Accelerator Facility has made the first direct measurement of the weak charge of the proton through the precision measurement of the parity-violating asymmetry in elastic electron-proton scattering at low momentum transfer. The electron-proton scattering rate largely depends on the five beam parameters: horizontal position, horizontal angle, vertical position, vertical angle, and energy. Changes in these beam parameters when the beam polarization is reversed create false asymmetries. 
Although attempt has been made to keep changes in beam parameters during reversal as small as possible, it is necessary to correct for such false asymmetries. 
To make this correction precisely, a beam modulation system was implemented to induce small position, angle, and energy changes at the target to characterize detector response to the beam jitter.
Two air-core dipoles separated by $\sim$10~m are pulsed at a time to produce position and angle changes at the target, for virtually any tune of the beamline. The beam energy was modulated using an SRF cavity. 
The hardware, associated control instrumentation will be described in this dissertation. Preliminary detector sensitivities were extracted which helped to reduce the width of the measured asymmetry. The beam modulation system also has proven valuable for tracking changes in the optics, such as dispersion at the target and beam position coupling.

There is a parity conserving Beam Normal Single Spin Asymmetry or transverse asymmetry ($B_{n}$) on $H_{2}$ with a sin($\phi$)-like dependence due to two-photon exchange. The size of $B_{n}$ is few ppm, so a few percent residual transverse polarization in the beam, in addition to potentially small broken azimuthal symmetries in the detector, might lead to few ppb corrections to the Q-weak data. As part of a program of $B_{n}$ background studies, we made the first measurement of $B_{n}$ in the N-to-Delta transition using the Q-weak apparatus. $B_{n}$ from electron-nucleon scattering is also a unique tool to study the $\gamma^{*}\Delta\Delta$ form factors. 
This dissertation presents the analysis of the first measurement of the beam normal single spin asymmetry in inelastic electron-proton scattering at a $Q^{2}$ of 0.0209 (GeV/c)$^{2}$. This measurement will help to improve the theoretical models on beam normal single spin asymmetry and thereby our understanding of the doubly virtual Compton scattering process.

%A robust, new strategy is presented for beam modulation for parity violating experiments at Jefferson Lab. The $Q_{weak}^p$ experiment will measure the parity violating asymmetry (-234 ppb) in elastic electron-proton scattering to determine the proton's weak charge with 4\% total uncertainty. The electron-proton scattering rate depends in first order on the five beam parameters: horizontal position (X), horizontal angle (X$^\prime$), vertical position (Y), vertical angle (Y$^\prime$), and energy (E). Small changes in these parameters will create a change in rate which, if beam helicity dependent, would create a false asymmetry. While the source group tries to keep these helicity-correlated parameter changes as small as possible, our beam modulation group will measure the detector sensitivities for later correction of beam false asymmetries. In order to measure the detector sensitivities, we will modulate X and X$^\prime$ with a pair of coils, Y and Y$^\prime$ with another pair of coils, and the beam energy using an SRF cavity. This will be done at frequencies of O(100 Hz) to roughly match the planned helicity-reversal rate. We have estimated the modulation magnitudes needed to obtain 10\% errors on the sensitivities every day while consuming only 1\% of our beamtime. 
%Each pair of coils, separated by 9.5 meters, will be simultaneously pulsed with a specific ratio of currents determined from OPTIM to produce nominally pure position and angle changes at the target. We find that angle changes require field integrals and orbit deviations which are an order of magnitude larger than position changes yielding a similar false asymmetry. Our design is robust in the sense that, should the 3C beamline optics ever change, we can simply modify the ratio of coil currents without having to move coils. We have thoroughly tested a coil on the bench powered with a JLab TrimII power amplifier and shown that sinusoidal operation up to ~250 Hz of the required magnitude will be possible at the nominal Qweak beam energy of 1.165 GeV.  Issues with extending this system other PV experiments in Hall C and Hall A are discussed. 