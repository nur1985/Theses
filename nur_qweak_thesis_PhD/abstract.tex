
The Q-weak experiment in Hall-C at the Thomas Jefferson National Accelerator Facility has made the first direct measurement of the weak charge of the proton through the precision measurement of the parity-violating asymmetry in elastic electron-proton scattering at low momentum transfer. 
There is also a parity conserving Beam Normal Single Spin Asymmetry or transverse asymmetry ($B_{n}$) on $H_{2}$ with a sin($\phi$)-like dependence due to two-photon exchange. If the size of elastic $B_{n}$ is a few ppm, then a few percent residual transverse polarization in the beam, combined with small broken azimuthal symmetries in the detector, would require a few ppb correction to the Q-weak data. As part of a program of $B_{n}$ background studies, we made the first measurement of $B_{n}$ in the N-to-$\Delta$(1232) transition using the Q-weak apparatus. 
The final transverse asymmetry, corrected for backgrounds and beam polarization, was found to be $B_{n}$ = 42.82 $\pm$ 2.45 (stat) $\pm$ 16.07 (sys)~ppm at beam energy $E_{beam}$ = 1.155~GeV, scattering angle $\theta$ = 8.3\degrees{}, and missing mass $W$ = 1.2~GeV.
%This dissertation presents the analysis of the first measurement of the beam normal single spin asymmetry in inelastic electron-proton scattering at a $Q^{2}$ of 0.0209 (GeV/c)$^{2}$. 
$B_{n}$ from electron-nucleon scattering is a unique tool to study the $\gamma^{*}\Delta\Delta$ form factors, and this measurement will help to improve the theoretical models on beam normal single spin asymmetry and thereby our understanding of the doubly virtual Compton scattering process.


%The electron-proton scattering rate largely depends on the five beam parameters: horizontal position, horizontal angle, vertical position, vertical angle, and energy. Changes in these beam parameters when the beam polarization is reversed create false asymmetries. Although attempt has been made to keep changes in beam parameters during reversal as small as possible, it is necessary to correct for such false asymmetries. 

To help correct false asymmetries from beam noise, a beam modulation system was implemented to induce small position, angle, and energy changes at the target to characterize detector response to the beam jitter.
Two air-core dipoles separated by $\sim$10~m were pulsed at a time to produce position and angle changes at the target, for virtually any tune of the beamline. The beam energy was modulated using an SRF cavity. 
The hardware and associated control instrumentation will be described in this dissertation. 
Preliminary detector sensitivities were extracted which helped to reduce the width of the measured asymmetry. The beam modulation system has also proven valuable for tracking changes in the beamline optics, such as dispersion at the target.

