%chapdiscussion.tex
%
\Section{Conclusions}
\label{Conclusion}
The nuclear transparency of kaons in the reaction $A(e,e'K^+)$ was measured by forming a super ratio of the experimental yield to the Monte Carlo simulation yield from $^{12}$C, $^{63}$Cu and $^{197}$Au to the yields from $^{2}$H at $Q^2$ = 1.1, 2.2 and 3.0 $(\mathrm{GeV/c})^2$. 
%Nuclear transparency was by formed a super ratio of the experimental yield to the Monte Carlo simulation yield from given targets with nucleon number $A$ and deuterium. 
%We found reasonable agreement between the experimental and Monte Carlo distributions. %Both the energy dependence and the $A$ dependence of the transparency show slight deviations from the traditional nuclear physics expectations. The energy dependence of effective cross-section from this experiment for electroproduction of kaons is consistent with the existing world data of kaon-nucleon cross-section. We can not claim the inconsistency of cross-section dependence on $\alpha$ like other experiments of Jefferson Lab for proton and pion with theoretical prediction \cite{carroll, PhysRevLett.87.212301} for low statistics.
Both the energy and $A$ dependence of the nuclear transparency are consistent with traditional nuclear physics expectations within experimental errors. 

The effective kaon-nucleon cross sections extracted from the nuclear transparency are also consistent with the world data on hadron-scattering. The electromagnetic interaction is weaker than the strong interaction, hence electrons can probe deeper into the nucleus compared to hadrons. This manifests as a scale factor between the effective hadron-nucleon cross sections extracted from electro-production vs hadro-production measurments. The magnitude of these scale factors are an excellent measure of the differences in the propagation of nucleons vs mesons in the nuclear medium.
%$\alpha$ from electro-production of protons and pions deviate from hadron-scattering reaction. Due to large statistical uncertainties we can not realy confirm the discrepancy or consistency of $\alpha$ for electro-production of kaons with the hadron-scattering reaction.

For protons and pions, the $A$ dependence of the nuclear transparency, (parameterized as $\alpha$), extracted from electro-production  measurements do not agree with those obtained from hadro-production. Some of these decrepancy can be also explained in terms of the differences in the probes used.
%However, given the large statistical uncertainties, our results do not really constrain the expectations.

For kaons,  the alpha extracted from our elecro-prodction data are consistent within statistical uncertainties, with the hadro-production results. However, given the large statitical uncertainties of our results, we are unable to draw any strong conclusions. 

This analysis adds to our understanding of kaon propagation in the nuclear medium and will help improve the design of future kaon electro-production experiments aimed at studying the reaction mechanisms.
%The final result for nuclear transparency and its  and $A$ dependence has been shown \figureref{transp2}. We also calculated nucleon-nucleon effective cross section and has been shown in the \figureref{crr2}. The parameter $\alpha$ was shown as a function of effective cross section in \figureref{carolplot}.%The results suggest slight enhancement of the nuclear transparency as a function of $Q^2$ and $P_K$ which is reasonable agreement with theoretical predictions of color transparency  within the uncertinity. 
%These data do not provide any conclusive evidence for the color transparency effect \cite{PhysRevC.74.018201, PhysRevD.62.113009} for similar reason.

\Section{Further Research}%
In the future, JLab will upgrade it's energy from 6 GeV to 12 GeV and the nuclear transparency effect will be probed at higher $Q^2$ compared to previous experiments. Experiment E12-06-107 aims to confirm the onset of CT in pions and search for CT in protons. Studies of hadronization in nuclei by deep inelastic scattering and L-T separated kaon electro-production cross section from 5-11 $GeV$ will be performed by experiments E12-07-101 and E12-09-011, respectively. We hope these experiments will give us a better understanding of the reaction mechanism of kaon electro-production and kaon propagation through nuclei.
%In the future JLab will upgrade it's energy from 6 $GeV$ to 12 $GeV$ and the nuclear transparency effect will be probed at larger $Q^2$ and $P_k$ compared to $k$CT, where the enhancement of the nuclear transparency is expected to be larger. E12-06-107 experiment in Jefferson Lab will search for color transparency at 12 $GeV$. Studies of hadronization in nuclei by deep inelastic scattering and L-T separated kaon electroproduction cross section from 5-11 $GeV$ will also be performed in E12-07-101 and E12-09-011 experiments  respectively. We hope these experiments will give us a better understanding of kaon transparency from electroproduction.