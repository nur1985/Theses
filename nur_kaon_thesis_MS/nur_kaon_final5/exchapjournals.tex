%exchapjnls.tex Features for Compatibility with Journals
%
This chapter explains a variety of commands defined by {\tt msuthesis.cls}
that you can redefine when your are preparing an article for
a journal.  
Thus, your writing ({\tt *.tex}) will need few modifications
to conform to the journal's style requirements.
Some journals provide \LaTeX\ style files, as well.

\Section{Organize Your Files for Journal Articles}%
%
{\sloppy
You may be able to publish the content of individual chapters
of your thesis or dissertation
as journal articles.  
Note that journal articles do not have chapters.
To allow for possible publication,
put all \verb+\chapter+ commands in your main thesis file
(\eg {\tt thesis.tex}),
and let \verb+\Section+ be the highest level in any chapter file.

} %end sloppy

\Section{Citations}%
%
{\tt msuthesis.cls} provides commands 
that combine punctuation and citation.
For example, this citation%
\spacecite{SPTD07} is in the middle of a sentence.
This citation%
\commacite{SPTD07} is combined with a comma,
and this citation is combined with a period%
\periodcite{SPTD07}
Journals have various styles of citations which differ in 
where spaces and punctuation should go.
Your thesis may be one style and the final submission
of a journal article
may be a different style.
Use the following commands 
so that changes are easy to make.

\newspacing{\singlespacing}\begin{verbatim}
\spacecite{biblabels} %in the middle of a sentence
\commacite{biblabels} %includes a comma
\periodcite{biblabels}%includes a period
\end{verbatim}\newspacing{\defaultspacing}

\Section{References}%
%
\nocite{*}%
The following are examples in the References and guidance 
from the IEEE Computer Society.

\begin{itemize}
\item
Article in a collection \cite{Albrecht81:PPIFTE}
\item
Article in a conference proceedings \cite{Weiser98:ICDE}
\item
Article in a journal or magazine \cite{SSS74:ACMCS}
\item
Book \cite{Arbib98,NS79}
\item
\textsc{cd-rom} \cite{NS79:CDROM}
\item
Dissertation or thesis \cite{Fagin87:PhD,Nichols85:MS}
\item
Electronic publication \cite{Burka00:MUD}
\item
Newsletter \cite{Butler87:SDN}
\item
Non-English source, 
when original title is included \cite{ZS84},
and when original title is unprintable \cite{Nakayama95:NE}
\item
Personal communication and
                 unpublished materials
are usually not referenced because they are not
                 available to the reader. 
However, footnotes can be used to give credit to such sources.
General thanks can be given in an acknowledgments section.

\item
Standard \cite{IEEE15961992,BT60192}
\item
Technical memo \cite{IST83misc}
\item
Technical report with report number \cite{HH85:Cornell674}
and without report number \cite{Yarwood77:UToronto}
\item
Technical or user manual without authors \cite{UnixID86}
\end{itemize}

In references to electronic publications,
be sure to include all punctuation exactly as supplied (hyphens
and tildes ($\sim$), in particular, are very common in Web addresses).
Tilde is produced by \verb+$\sim$+ in \LaTeX.
Verify addresses you tag as \textsc{url}s by copying and pasting them
into your browser and seeing if the string of text that is in
your document actually goes where it should.
If the address must run across more than one line, follow these
guidelines:
   \begin{itemize}
   \item Break only after a forward slash or a dot (period). 
   \item Do not split the double slash. 
   \item Do not split at hyphens, tildes, and so on, 
         that are part of the address. 
   \item Do not introduce hyphens to break words. 
   \item Separating the extension (for example, the html at the
                    end) is discouraged. 
   \end{itemize}
Some examples using\\
\verb+http://www.web-pac.com/mall/pacific/start.html+:

\noindent Acceptable:
\begin{verbatim}
http://
www.web-pac.com/mall/pacific/start.html 

http://www.web-pac.
com/mall/pacific/start.html

http://www.web-pac.com/mall/
pacific/start.html
\end{verbatim}
Not acceptable:
\begin{verbatim}
http:/
/www.web-pac.com/mall/pacific/start.html

http://www.web-
pac.com/mall/pacific/start.html

http://www.web-pac
.com/mall/pacific/start.html

http://www.web-pac.com/mall/paci-
fic/start.html
\end{verbatim}
Discouraged:
\begin{verbatim}
http://www.web-pac.com/mall/pacific/start.
html
\end{verbatim}


In titles and captions,
capitalize the first and last words, and all nouns, pronouns,
adjectives, verbs, adverbs, and subordinating conjunctions.
Lowercase articles, coordinating conjunctions, and
prepositions, regardless of length. Example: Toward Better
Real-Time Programming through Dataflow. 

To make a source easy for researchers to find, use the title as
it originally appears. Do not add or remove hyphens, change
words to preferred spellings, or lowercase internal capitals. 

Do not include the editor's name for a conference
proceedings unless it is a carefully edited volume published
as a regular book.

The IEEE Computer Society's Web page also has detailed advice
regarding references to electronic publications and other matters.%
\footnote{%
See {\footnotesize\tt http://computer.org/author/style/refer.htm}. Click Special Sections; click References.  
}

\Section{Common Abbreviations}%
%
{\tt msuthesis.cls} provides commands to produce abbreviations of 
common Latin expressions for compatibility with various journals.
Some journals require italics and some require Roman fonts.  
The abbreviations are: \etc, \eg, \ie, \etal, and \vs
Various publishers prefer different fonts for certain
abbreviations.  The following commands are available
to provide flexibility
when a thesis chapter is published as a journal article.

\newspacing{\singlespacing}\begin{verbatim}
\etc
\ie
\eg
\etal
\vs
\end{verbatim}\newspacing{\defaultspacing}

\noindent Note that periods after abbreviations are included
inside each command.


\Section{Equations}%
%
Don't start a paragraph with a displayed equation, nor make
one a paragraph by itself.

You should use \verb+\begin{equation}+ and \verb+\end{equation}+
to define one equation so that equation numbers are generated,
instead of \verb+\[+ and \verb+\]+.
It is convenient to use
\verb+\begin{eqnarray}+ and \verb+\end{eqnarray}+ 
to define all equations
just in case a long equation needs to be broken into
multiple lines later.

Each major equation should have a \verb+\label{eq:label}+ command
at the end of the equation.
Journals have various standards for referring 
to equation numbers in paragraphs.
The following commands are provided by {\tt msuthesis.cls}
as an alternative to 
the standard \LaTeX\ command \verb+Equation~(\ref{eq:label})+.

\newspacing{\singlespacing}\begin{verbatim}
\eqnref{lable}         %in middle of sentence
\eqnsref{firstlabel}   %in middle of sentence
\andeqnref{lastlabel}  %at end of list
\Eqnref{label}         %at start of sentence
\Eqnsref{firstlabel}   %at start of sentence
\end{verbatim}\newspacing{\defaultspacing}

If you are referring to one equation, use
\verb+\eqnref{label}+,
or as the first word in a sentence, \verb+\Eqnref{label}+.
If you are referring to a list of equation numbers, use

\newspacing{\singlespacing}\begin{verbatim}
\eqnsref{firstlabel}, \ref{eq:secondlabel}, 
\andeqnref{thirdlabel}
\end{verbatim}\newspacing{\defaultspacing}

\noindent 
or similarly, at the beginning of a sentence,
\verb+\Eqnsref{firstlabel}+ \etc
\Eqnref{example} is an example of a multiline equation.
%
\begin{eqnarray}
y &=& a x + b
\nonumber\\
&+& c x + d
\label{eq:example}
\end{eqnarray}



\Section{Theorems}%
%
The following theorem environments are defined by 
{\tt msuthesis.cls}
for compatibility with journals 
published by Elsevier and Kluwer.

\newspacing{\singlespacing}\begin{verbatim}
\begin{thm} \end{thm} %Theorem
\begin{lem} \end{lem} %Lemma
\begin{cor} \end{cor} %Corollary
\begin{defn}\end{defn}%Definition
\begin{case}\end{case}%Case
\begin{pf}  \end{pf}  %Proof
\end{verbatim}\newspacing{\defaultspacing}

The command \verb+\qed+ is defined by {\tt msuthesis.cls}
to produce a black box;
it is used by the \verb+pf+ environment at the end of proofs.

Unfortunately, the University's {\em Standards} prohibits using bold and italics for things like theorem headings and text.


The following are examples of these.

\begin{thm}
This is an example theorem.  
\end{thm}
\begin{pf}
This is a proof.
\end{pf}

\begin{lem}
This is an example lemma.
\end{lem}
\begin{pf}
This is another proof.
\end{pf}

\begin{cor}
This is an example corollary.
\end{cor}

\begin{defn}
This is an example definition.
\end{defn}

\begin{case}
This is an example case.
\end{case}

\begin{case}
This is another case.
\end{case}

\Section{Tables}%
%
{\sloppy 
If a table needs to span the full page 
in a double-column publication,
use the \verb+\Tabledbl+ command. In a single-column thesis,
it looks the same as the \verb+\Table+ command.

} %end sloppy

Journals have various standards for referring
to table numbers in paragraphs.
The following commands are defined by {\tt msuthesis.cls}
as alternatives to 
the standard \LaTeX\ command \verb+Table \ref{tab:filename}+.

\newspacing{\singlespacing}\begin{verbatim}
\tableref{filename}  %in middle of sentence
\Tableref{filename}  %as first word of sentence
\end{verbatim}\newspacing{\defaultspacing}

\noindent where {\tt filename} is the identifier for your table data.


\Section{Figures}%
%
Some journals and conference proceedings are published 
in two column format.  Some figures fit within one column, 
but others must span both columns (usually at the top of a page.  
The following command
is used when you want the figure to span two columns
if the document is in two-column format.
If this command is used in a one column document,
such as a thesis, 
then the figure is typeset the same as 
the ordinary \verb+\Figure+ command.

\newspacing{\singlespacing}\begin{verbatim}
\Figuredbl{filename}{\figwidth}{This is a Caption}
\end{verbatim}\newspacing{\defaultspacing}

This command lets your thesis chapter's {\tt *.tex} file 
be practically the same
as your conference paper's with respect to figures.

Journals have various standards for referring 
to figure numbers in paragraphs.
The following commands are defined by {\tt msuthesis.cls}
as alternatives to 
the standard \LaTeX\ command \verb+Figure~\ref{fig:filename}+.



\newspacing{\singlespacing}\begin{verbatim}
\figureref{filename} %in middle of sentence
\Figureref{filename} %as first word of sentence
\end{verbatim}\newspacing{\defaultspacing}

\noindent where {\tt filename} is the identifier for your artwork.

