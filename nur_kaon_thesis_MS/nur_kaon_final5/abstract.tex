% abstract.tex
%\begin{center}
%\newpage
%ABSTRACT
%\end{center}

Hadron propagation in the nuclear medium is essential for building an accurate model of the nuclear many-body system. Quasi-free electron scattering from nuclei is one of the tools used in the study of hadron propagation effects in the nuclear medium. Electro-production and propagation of kaons from nuclei provides an additional (strangeness) degree of freedom, inaccessible with other hadrons. An experiment to measure the transparency of pions was completed in Dec 2004 at the Thomas Jefferson National Accelerator Facility in Hall C. Using same data set, we report the first measurement of the nuclear transparency of kaons for $^{12}$C, $^{63}$Cu and $^{197}$Au nuclei at $Q^2$= 1.1, 2.0 and 3.0 $(\mathrm{GeV/c})^2$. We have also extracted the average effective kaon-nucleus cross section from the nuclear transparency. The $Q^2$ and $A$ dependence of the transparency and average effective cross section are compared to results from kaon-nucleus scattering data and found consistent within experimental uncertainties.
