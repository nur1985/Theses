%exfaq.tex Frequently Asked Questions
%
\Section{How can I remove DRAFT on the title page?}%
The third line of {\tt examplethesis.tex} is

\newspacing{\singlespacing}\begin{verbatim}
\usepackage[light,first]{draftcopy} 
%remove for final version
\end{verbatim}\newspacing{\defaultspacing}

Putting a \% at the beginning of this line will make this 
line a comment, disabling DRAFT.

\Section{Why can't \LaTeX\ find my files?}%
%
You must put all files for your document in the same directory,
including all the figures (encapsulated postscript {\tt *.eps}). 
A warning in the {\tt *.log} file may indicates that you had a command 
like \verb+\Figure{./Drawings/tiles_fuel}{+\ldots.
The template cannot handle paths. Remove paths in all 
\verb+\Figure+,
\verb+\Table+, 
and \verb+\input+ commands.

\Section{How can I fix the top margin of my title page?}%
%
If your title is 1 or 2 lines, then the top margin of the title page should be 
satisfactory, 3.5--4.0 inches.

If your title is 3 lines, the the top margin will be a bit more than 3 inches
which does not conform with the University's standard.  
The library has agreed to waive the 3.5 inch rule for theses and dissertations
formatted with this template (per James Nail, 6/12/09).

If you use this template, you should inform the library staff of this fact 
when you turn it in for review.


\Section{Why is there extra white space after a section title?}%
%
In most cases, 
you should use \verb+\Section+, not \verb+\section+.  
Also note that there must be no blank lines between a 
\verb+\Section{}+ command and the the paragraph text, 
\eg
\verb+\Section{bla bla}This begins a paragraph+\ldots
This rule was necessary to assure that 
the first paragraph of a section is indented, 
as required by the university.

\Section{How do I create subsubsubsection 3.5.3.1.1?}%
%
The template does not support subsubsubsections. 

They are usually short and can often be included in the 
higher-level subsubsection.  
A sentence or two could make the transition for the
reader in lieu of a subsubsubsection title.

An alternative workaround is to use an {\tt enumerate} environment
to create a numbered list.  See your \LaTeX\ book for details.


\Section{Why are figure titles double spaced instead of single?}%
%
This template does not support figure captions, table captions, 
or section titles that are more than one line
in the table of contents, 
list of figures, 
or list of tables.  
The solution is to shorten the titles to less than one line.  
The extra information removed from a caption can be
annotated inside the figure or table, 
or can be in an ordinary paragraph that explains the figure or table.

Some journals use a block-caption style, where the caption explains the figure or table.  This template does not support block-caption style, because the University's requirements for the List of Figures and List of Tables are incompatible with this style.

\Section{How do I fix not enough memory for figures and tables?}%
%
A large number of figure/tables
on consecutive pages causes \LaTeX\ to become confused due 
to limited memory.  When you have numerous consecutive ``floats'' 
(\ie figures or tables), you should have a \verb+\clearpage+ command
between groups of three to five floats.   
{\tt Clearpage} forces a page break and typesets all known floats
beginning on the new page.  So you should position the {\tt clearpage}
commands at points you want a page break to occur, 
such as between consecutive figures.

\Section{How can I refer to a list of equations, tables, or figures?}%
See Sections~4.5, 4.7, and~4.8.
Here are three examples of lists of references.

\newspacing{\singlespacing}\begin{verbatim}
\eqnsref{aaa}, \ref{eq:bbb}, \andeqnref{ccc}
Tables~\ref{tab:aaa}, \ref{tab:bbb}, and~\ref{tab:ccc}
Figures~\ref{fig:aaa}, \ref{fig:bbb}, and~\ref{fig:ccc}
\end{verbatim}\newspacing{\defaultspacing}

\Section{How can I make a table heading span more than one column?}%
See your \LaTeX\ book for commands associated 
with the tabular environment, 
such as \verb+\multicolumn+.

\Section{Why is a missing \$ inserted?}%
%
When your {\tt *.log} file has this warning,
{\tt ! Missing \$ inserted.},
this is most often caused by the use of underline characters in 
nonmath mode.  
If you have a filename, URL, \etc that has an underline in the
middle of a paragraph, you must substitute \verb+\_+ 
for the plain underline.  

Plain underline is interpreted as a
subscript command in math mode. 
\LaTeX\ assumes you want to begin math mode.

\Section{How can I typeset ``John Wiley \& Sons'' in a reference?}%
BibTeX fields can have \LaTeX\ commands like paragraphs.  
Ampersand is a special character.
See your \LaTeX\ book for escape sequences for special characters, 
such as \verb+\&+.

\Section{How can I typeset a tilda in a URL?}%
Tilda is a special character in \LaTeX.
A ``similar'' symbol ($\sim$) looks good in URLs.  
It is created by \verb+$\sim$+.  See Section 4.3.

\Section{How can I change to single spacing?}%
See Section~3.3.

\Section{How can I change to bold and then back to normal print?}%
{\sloppy
See your \LaTeX\ book for commands associated with fonts,
such as \verb+\bfseries+,  
\verb+\mdseries+, 
and \verb+\textbf+.

} %end sloppy

\Section{How can I create a superscript or subscript?}%
See your \LaTeX\ book for math mode formatting symbols,
such as \verb+^+ and \verb+_+.

\Section{How can I make a fraction in an equation?}%
See your \LaTeX\ book for math mode commands, 
such as \verb+\frac+ in displayed equations.
Stacked fractions are ugly inline in paragraphs; 
use / instead to represent division in paragraphs.

\Section{How can I typeset all my references?}%
{\sloppy
Even though \LaTeX\ provides a \verb+\nocite+ command,
which causes all references in the {\tt *.bib} file to be in the References section,
I recommend that you just put a \verb+\cite+ command 
for each reference in any convenient place in your draft document.  
Later you can add the sentences to go with each citation.
Because the university requires that all references be cited,
your final thesis should not use 
\verb+\nocite+ at all.

} %end sloppy

