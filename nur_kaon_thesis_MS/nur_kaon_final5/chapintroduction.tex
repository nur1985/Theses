%chapintroduction.tex
%

\Section{Experimental Motivation}%
%
\label{Experimental Motivation}
The propagation of hadrons in the nuclear medium is an essential element of the nuclear many-body problem, which seeks to better understand how to construct the many-body nuclear system from the more basic meson-nucleon and nucleon-nucleon amplitudes. Quasi-free electron scattering from nuclei provides an excellent tool for a microscopic examination of hadron propagation effects in the nuclear medium. The relative weakness of electromagnetic probe and very well understood interaction at the production vertex are some of the well known advantages of electron scattering. Hence, quasi-free production can be viewed as tagging a source of hadrons emerging from throughout the nuclear volume, with minimal disruption of the system. Thus over the past two decades, proton propagation in nuclei has been studied extensively using quasi-free electron scattering \cite{bates,ne18,ON95,Abb95,jlabp2}. Recently, pion propagation in nuclei has also been studied using exclusive electroproduction of pions from nuclei \cite{jlabpi1,jlabpi2}. 

Electro-production\footnote{Electron scattering reaction, where electrons are used as a probe.} of $K^+$ from nuclei provides an additional (strangeness) degree of freedom, which is inaccessible with nucleons and pions. Strangeness provides a unique window on the nuclear many-body problem via access to energy levels that protons and neutrons cannot occupy. Moreover, the $K^+$-nucleon ($K^+-N$) interaction is relatively weak and varies smoothly with energy \cite{dover77}, which makes them ideal for studying the $K^+$-nucleus interaction, as well as the physics of hadron formation. Typically, the $K^+$-nucleus interaction has been studied using $K^+$ scattering from nuclear targets, and these experiments can be considered as analogous to electron scattering since both involve weakly interacting probes.  

In spite of the $K^+-N$ interaction being relatively weak and free of resonance structure, it is difficult to build a description of the kaon-nuclear scattering from the elementary $K^+-N$ interaction \cite{skg1}. Differential and total cross section measurements \cite{kccexpt1,kccexpt2,kccexpt3,kccexpt5,kccexpt6} from  $K^+$ scattering experiments show significant discrepancy when they are compared to theoretical calculations \cite{skg1,kccth1,kccth2,kccth3}. Even in experiments where the ratio of total cross section for a nuclear target to that for a deuteron is measured \cite{kccexpt1,kccexpt2,kccexpt3,kccexpt5,kccexpt5,kccexpt6}, where the theoretical uncertainties are expected to cancel in the ratio, the theoretical results are found to be systematically smaller than the data \cite{skg1, kccth1,kccth2,kccth3}.

As conventional nuclear physics models have failed to account for the data, several exotic mechanisms have been proposed by various authors \cite{ernst95}. These include modification of the nucleon size in the nuclear medium \cite{skg1}, reduced meson masses in the nuclear medium \cite{kccth4}, meson exchange currents \cite{kccth5,kccth6}, long range correlations \cite{kccth7}, and various other mechanisms \cite{kccth8}. 

\Section{Nuclear Transparency}%
The electro-production of kaons from nuclei is an excellent alternative to the kaon-nucleus scattering in order to extract nuclear transparency. Nuclear transparency is defined as the ratio of the cross section per nucleon for a process on a bound nucleon in the nucleus to the cross section for the process on a free nucleon \cite{Dutta:2004kw}. One can parametrize the hadron nucleus cross section as $\sigma_N$ = $\sigma_0 A^{\alpha}$, where $\sigma_0$ is the N-N cross section in free space. Then nuclear transparency can thus be parametrized as T = $A^{(1-\alpha)}$. In other words, nuclear transparency can be interpreted as the probability that the hadron produced in the reaction is not scattered outside of the experimental acceptance by the residual nucleus.
%An experiment to measure the transparency of \textcolor{red}{pions}, in search of CT was completed in Dec 2004 at JLab in Hall C through the elctro-production of pions by the reaction A(e,$e^\prime \pi^+$). The same set of data also has a considerable sample of kaons that can be used to study the transparency of kaons.\\
%My first job was to seperate kaons from the pions and protons by applying different Particle IDentification (PID). Once I have kaons seperated, then I can measure the cross-section. But, before corss-section measurement I need to compare the data with simmulation model from Hall-C called SIMC for several variables in order to check whether we choosed kaons properly or not. Now we have all the cross-sections for Hydrozen and other targets, then we can extract trnaparecy.\\

\Section{Previous Measurements}%
%
There were several experiments performed in last decade at Thomas Jefferson National Accelerator Facility (Jefferson Lab) to measure the transparency of kaons from different processes. The study of $(e,e^\prime K^+)$\footnote{Reaction represents electron scattering from nucleon or nuclear target with final product as scattered electron and kaon.} on carbon and aluminum were performed in Jefferson Lab during 2001. The quasi-free production of the $\Lambda$, $\Sigma^0$ and $\Sigma^-$ hyperons was studied in the reaction. An $A$ dependence of effective nucleon number was analyzed and the cross section was also fitted to a power law \cite{hinton01}. Another experiment E93-018 at Jefferson Lab was performed to measure kaon electro-production on hydrogen in two hyperon channels $(e,e^\prime K^+)\Lambda$ and $(e,e^\prime K^+)\Sigma$. This data was taken for $Q^2$ = 0.52, 0.75, 1.00 and 2.00 $\mathrm{(GeV/c)^2}$ in this experiment. Cross sections averaged over the azimuthal angle were extracted at each of these twelve points for each hyperon. Rosenbluth separations were performed to separate the longitudinal and transverse production cross sections \cite{RM99}. The quasi-elastic $(e,e^\prime p)$ reaction was studied on targets of deuterium, carbon, and iron up to a value of momentum transfer $Q^2$ of 8.1 $\mathrm{(GeV/c)^2}$ at Jefferson Lab. A nuclear transparency was determined by comparing the data to calculations in the experiment. The dependence of the nuclear transparency on $Q^2$ and the mass number $A$ was investigated \cite{jlabp2}. 
%
%Several other experiments were performed to search for color transparency at Brookhaven National Laboratory (BNL) \cite{PhysRevLett.61.1698,PhysRevLett.81.5085, PhysRevLett.87.212301} and at Jefferson Lab \cite{PhysRevLett.61.1698, BC06}.
%The first experiment designed to search for color transparency performed at Brookhaven National Laboratory (BNL) \cite{PhysRevLett.61.1698} in the late 1980s using the $^{12}C(p,2p)$ reaction. More measurements of the nuclear transparency were performed at BNL using the same reaction \cite{PhysRevLett.81.5085, PhysRevLett.87.212301}. The nuclear transparency was defined as the cross section for elastic p p scattering in the nucleus divided by the cross section for elastic p p scattering in hydrogen, with corrections for Fermi motion of the proton in the nucleus \cite{PhysRevLett.61.1698, BC06}. The observed nuclear transparency increased as a function of the beam energy and then decreased and this behavior was not predicted and explained by traditional nuclear physics calculations. 
%The nuclear transparency was observed to be energy independent from $Q^2$ = 2 $(GeV/c)^2$ to the maximum measured $Q^2$ of 8.1 $(GeV/c)^2$ from deuterium, carbon, iron and gold targets. These measurements indicated that there was no significant effect from color transparency in the $A(e,e^\prime p)$ reaction up to $Q^2$ = 8.1 $(GeV/c)^2$. The absence of the color transparency effect in the above reaction has been interpreted as an indication that the proton formation length may only have been as large as internucleonic distances, rather than the size of the nucleus, in these experiments at BNL and Jefferson Lab \cite{PhysRevC.73.044003}.

\Section{Experimental Background}%
The electro-production of kaons from nuclei is an excellent alternative to the kaon-nucleus scattering in order to explore the propagation of kaons through the nuclear medium. It may help to resolve discrepancies between theory and experiment which are related to the details of the reaction mechanism or due to the various approximations made in calculating the cross sections. In this analysis, we report the first measurement of the nucleon number, $A$, and $Q^2$ dependences of nuclear transparency for the $A(e,e'K^+)$ process. The measurement was performed on $^{2}$H, $^{12}$C, $^{27}$Al, $^{63}$Cu and $^{197}$Au nuclei over a  $Q^2$ range of 1.1 $\mathrm{to}$ 3.0 $(\mathrm{GeV/c})^2$.

%In addition to the university requirements above, the
%Department of Computer Science and Engineering
%requires that all Master's
%project reports, thesis proposals, and dissertation proposals also comply.  Even though the department may
%provide tools or templates to facilitate preparing a document,
%the student is fully responsible for assuring that the final
%document complies with the {\em Standards} and the
%departmental requirements reproduced below.
%The {\em Standards} leave certain issues to be decided by the
%``degree-granting unit,'' which in our case is the Department
%of Computer Science and Engineering. 
%Acceptable styles are selected by the
%candidate's degree-granting unit, with the approval of the
%Office of Graduate Studies \cite{SPTD07}.  The following specifies
%departmental style requirements for dissertations,
%theses, and project reports,
%as published on the department Web site.%
%\footnote{%
%See {\tt http://www.cse.msstate.edu}.
%}
%%
%\begin{enumerate}
%\item
%The baseline type size shall be 12 points \cite{SPTD07}. 
%Exceptions are explained in the {\em Standards}.
%\item
%The font style shall be one of the following \cite{SPTD07}: 
%   \begin{itemize}
%   \item Times 
%   \item New Century Schoolbook 
%   \item Palatino 
%   \item Bookman 
%   \end{itemize}
%    Exceptions are explained in the {\em Standards}. 
%\item
%The abstract may include a list of key words \cite{SPTD07}. 
%\item
%The Table of Contents, List of Tables, and List of Figures
%	shall have the style shown in the {\em Standards}
%	\cite{SPTD07}.
%\item
%Section headings and subheadings within a chapter shall be
%bold and left-justified \cite{SPTD07}.
%\item
%Chapter numbers may be roman or arabic numerals.
%\item
%Sections and subsections shall be numbered consecutively
%	within the next higher-level section/subsection. A section
%	number shall be formed by an arabic chapter numeral, a
%	period, higher-level section numbers separated by periods,
%	and finally the current subsection number.  For example
%	2.3.4 denotes Chapter 2, the third section within Chapter
%	2, and the fourth subsection within section 2.3.
%\item
%Text shall be full-justified \cite{SPTD07}. 
%\item
%Italics is required, rather than underline, for non-English,
%	titles of books and journals, and similar uses \cite{SPTD07}.
%\item
%Important equations shall be numbered consecutively within
%	chapters, enclosed by parentheses, and formed by an arabic
%	chapter numeral, a period, and the equation number \cite{SPTD07}.
%	For example, Equation (2.1) indicates the first equation
%	in Chapter 2.
%\item
%Table and figure captions shall be centered with title-style
%capitalization \cite{SPTD07}.
%\item
%Table and figure numbers shall be consecutive within chapters
%	and formed by an arabic chapter numeral, a period, and the
%	table/figure number \cite{SPTD07}.  
%	For example, Figure 2.1
%	indicates the first figure in Chapter 2 \cite{SPTD07}.
%\item
%Reference entries shall be consecutively numbered enclosed by
%	brackets, as is common in IEEE Computer Society
%	transactions and conference proceedings \cite{SPTD07}. Citations
%	shall have corresponding form.  For example [1] cites the
%	first entry in the references.
%\item
%Multiple citations in one place shall be enclosed in one set
%	of brackets, in ascending numerical order, delimited by
%	commas, for example [1,7,22,43].
%\item
%References shall follow the last chapter of text, rather than
%	at the end of each chapter
%	\cite{SPTD07}. If used, any appendices shall follow thereafter
%	\cite{SPTD07}.  
%	The heading shall be REFERENCES \cite{SPTD07}.
%\item
%Reference entries shall be sorted alphabetically by authors'
%names, and book title \cite{SPTD07}.
%\item
%The style of reference entries shall conform to  examples at
%	the Dept. of Computer Science and Engineering
%	Web site, which follows IEEE
%	Computer Society practices \cite{SPTD07}.
%\item
%Multiple appendices shall be labeled with capital letters, per
%	the {\em Standards} \cite{SPTD07}.  If there is only one
%	appendix it will not be designated by a letter.  However,
%	labels of sections, figures, tables and equations in the
%	appendix shall use A as the chapter prefix.  This will
%	facilitate unambiguous labeling.
%\end{enumerate}
%
%The references in this document conform to 
%departmental requirements
%and match IEEE Computer Society examples,%
%\footnote{%
%See {\footnotesize\tt http://computer.org/author/style/refer.htm}. Click Special Sections; click References.  (We prefer the Computer Society style rather than the IEEE Transactions Dept. style.)
%}
%except for the following details.
%\begin{itemize}
%\item
%Full journal and conference names are used instead of
%abbreviations.
%\item
%Full state names are used instead of abbreviations. 
%\item
%Conference paper entries show the sponsoring organization and
%	city of the conference, instead of the publisher of the
%	proceedings and publisher's city.
%\end{itemize}
%\Section{What is \LaTeX?}%
%\LaTeX\ is a typesetting system that is used primarily in academia.
%\LaTeX\ files are compatible with many academic publishers' typesetting systems. 
%For example, \LaTeX\ files are one of the preferred forms for final submission of articles to the various {\em IEEE Transactions}.
%\LaTeX\ has very strong capabilities for typesetting mathematics beautifully and for managing bibliographies. Even though, its user interface is not ``friendly'' compared to commercial word processors,suitable style files obviate concern for detailed formatting issues. Consequently, some students will find it attractive for typesetting dissertations, theses, proposals, and project reports.
%This guideline describes how to typeset a document that is compatible with the above requirements using \LaTeX\  and associated specialized style files. This document is itself an example of a master's thesis, so its source files can be used as the starting point for your document.

\begin{figure}[!tbp]
\begin{center}
\addtolength{\figwidth}{-0.5in}
  \includegraphics[trim = 0mm 30mm 0mm 0mm,clip,width=1.0\columnwidth]{reaction_fig1}
  \caption[The schematic kaon electro-production reaction.]{\label{fig:reaction_fig1}The schematic kaon electro-production reaction.\\\\ The azimuthal angle between the scattering and reaction planes with respect to the direction of the virtual photon.}
\end{center}
\end{figure}

%\setlength{\figwidth}{0.8\linewidth}
%\Figure{reaction_fig1}{\figwidth}{The schematic kaon electro-production reaction. The azimuthal angle between the scattering and reaction planes with respect to the direction of the virtual photon.}
%\Figure{reaction_fig1}{\figwidth}{The schematic kaon electro-production reaction.}

\Section{Kinematics}%
%
The basic reaction studied for the experiment can be written as 
\begin{equation}
e + A \rightarrow e^\prime + K^+ + X
\end{equation}
\noindent
where X represents other particles in the final state (for example, a neutron and the residual $A-1$ nucleons).
A schematic of the reaction is shown in \Figureref{reaction_fig1}. The variables are defined in the lab frame. The scattering plane is the plane which contains the three-momentum of the incident and scattered electron and the reaction plane is the plane which contains momentum of virtual photon($\vec{q}$) and the kaon momentum vector. $\theta_e$ is the electron scattering angle and $\theta_{pq}$ is the angle between the three-momentum of the virtual photon and the kaon. $\phi_{pq}$ is the angle between the scattering plane and the reaction plane.

The kaon electro-production cross section can be expressed as a function of $Q^2$, $W$ and $t$, where $q^2$ = -$Q^2$ is the four-momentum transfer squared and $W$ is the invariant mass of the virtual photon and the target, given by
\begin{equation}
Q^2 = 4EE^\prime \sin^2(\theta/2)
\end{equation}
\begin{equation}
W = \sqrt{M_A^2 + 2M_A\omega - Q^2}
\end{equation}
where $\omega$ (also called $\nu$) is the energy of the virtual photon = $E - E^\prime$ in lab frame. $E$ is energy of incident electron and $E^\prime$ is that of the scattered electron.The four-momentum square of the momentum transferred to the nucleon(s), $t$, is given by
\begin{equation}
t = (q-p_K)^2 = (E_K -\omega)^2 - \left|P_K\right|^2- \left|q\right|^2 +2\left|P_K\right|\left|q\right|\cos(\theta_{pq})
\end{equation}

\noindent
here $P_K$ and $E_K$ are the momentum and energy of the kaon, respectively.

The different hyperons produced in the reaction, shown in \figureref{reaction_fig1}(described in Section \ref{Missing Mass}), were $\Lambda^0$, $\Sigma^0$ and $\Sigma^-$. The cross section in terms of the five-fold differential cross section can be expressed as $\frac{d^5\sigma}{dE^\prime d\Omega^\prime_e d\Omega^*_K}$. The electro-production cross section can then be written in terms of the scattered electron energy $E^\prime$, electron lab frame solid angle $d\Omega_e^\prime = d\sin\theta_e d\phi_e$, and kaon C.M. solid angle $d\Omega_K^* = d\sin\theta_{qK}^* d\phi$. The model we used is discussed in Section \ref{Physics Model}.